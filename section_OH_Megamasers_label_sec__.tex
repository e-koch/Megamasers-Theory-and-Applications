\section{OH Megamasers}
\label{sec:OH}

Hydroxyl (OH) produces maser emission in its ground state (see Table \ref{subsub:rad_pump}). Two levels of splitting occur in this ground state: (1) the unclosed shell of electrons in the ground state gives rise to $\Lambda$-doubling of all rotational levels, and (2) hyperfine splitting occurs between the unpaired electron and the H atom in the nucleus. Thus, there are four transitions in this ground-state with frequencies of 1612, 1665, 1667, and 1712 MHz with LTE line strengths of 1:5:9:1 \cite{lo2005}. The inversion mechanism accounting for all four transitions is complicated \citep{Elitzur_1992} and the relative maser line strengths can change drastically depending on the source's environment \citep{lo2005}. Masering in all four levels is extremely rare, and OH mega-maser detections are most often seen in the main 1665 \& 1667 MHz transitions, with 1667 MHz emission often dominating for mega-masers. This is the opposite situation of galactic masers associated with HII regions, where the 1665 MHz line is the dominant feature in the spectrum. Supernovae remnants show significant emission in the 1712 MHz line. These differences point to differences in the environmental properties where the maser occurs. 

There are approximately 100 detected mega-masers to date (some detections have been called into question or cannot be adequately determined in the literature, see \citet{darling2002_paperIII}). The majority of these detections have come from large-scale surveys with explicit source selection. The most successful and complete of these surveys to date is that by \citet{darling_2000_paperI} with the Arecibo telescope, which resulted in 52 new OH mega-maser detections to $z<0.23$ \citep{darling2002_paperIII}. This survey, combined with previous detections over the previous few decades, have allowed the basic properties of the OH mega-masers and host galaxies to be fairly-well understood. These processes trace the environment within $\approx 100$ pc of the galactic nuclei in regions of extreme starbursts. These regions are typically highly obscured, making it difficult to study their nature at other wavelengths.  Current instrumentation is capable of detecting galactic strength masers out to $\approx 20$ Mpc \citep{darling2012}, potentially allowing for the discovery of many more OH mega-masers in the near future. Such studies could constrain many important properties of starbursts and galaxy mergers to significant redshifts.


\subsection{Extragalactic Properties}
\label{sub:oh_gal_props}

The dominance of 1665 MHz line emission over the 1667 MHz emission from OH mega-masers suggests they form in a different environment than OH galactic masers. Current detections support this, as nearly all OH mega-masers occur in luminous or ultra-luminous infrared galaxies (LIRGs/ULIRGs). These galaxies are unique since they harbour some of the most extreme starburst activity in the universe XXX CITE XXX, and are almost exclusively caused by major galaxy mergers \citep{clements1996}. In fact, LIRGs and ULIRGs are classified based on their far infrared (FIR) luminosities alone ($L_{\mathrm{FIR}} \ge 10^{11.4}$ L$_{\odot}$ for LIRGs, $L_{\mathrm{FIR}} \ge 10^{12}$ L$_{\odot}$).  The star-formation activity is centered around at least one of the nuclei in the system. OH mega-masers tend to be found within $\sim 100$ pc of this nucleus, where there is an abundance of heated dense molecular gas \citep{lo2005}. The extreme star formation rates near the nucleus lead to widespread heating of the dust and gas in the ISM through many giant HII regions, resulting in elevated emission in the IR. \citet{Elitzur_1992} note that IR radiation is the only pumping agent capable of permeating the large galactic volumes required. This connection of a dense, heated molecular gas environment to OH maser emission was recognized early on by \citet{Bottinelli_1987}, as such conditions are the observed locations of galactic OH maser emission. 

The relation between the far-infrared (FIR) luminosity and the isotropic OH maser luminosity was rigorously established using 95 detections by \citet{darling2002_paperIII}. They find a power-law relation of $L_{\mathrm{OH}} \propto L_{\mathrm{FIR}}^{1.2\pm0.1}$, after thoroughly correcting for the biases in their sample. This updated the initial relationship found by \cite{Baan_1989}, which had an index of 2.  This index fits within the expectation for a mix of saturated and unsaturated masers. \citet{darling2002_paperIII} show that high-gain, unsaturated masers are expected to have an index 2, while an index of 1 should result for low-gain saturated masers. This is related to the source of the stimulated emission: the unsaturated case stimulates the background radio continuum, which itself is correlated with the FIR radiation \citep{Yun_2001}. Furthermore, this result establishes that, despite differing from galactic OH maser emission, OH mega-masers arise due to radiative pumping. This relationship suggests that if even brighter galaxies than the known ULIRGs are discovered, one may find a class of {\it giga}-masers, particularly if mega-maser detections out to $z\sim 2$ can be detected.  Two such systems were detected by \citet{darling2002_paperIII}.  The properties of OH mega-maser hosts are more thoroughly explored in \S\ref{sub:oh_mergers}.

There are two observed components of OH mega-masers, diffuse emission on $\sim 100$ pc scales, and parsec scale compact regions. Both of these components appear to only require fairly ordinary conditions \citep{lo2005}. The diffuse component occurs from low-gain, unsaturated masers which amplify continuum emission from the heated, diffuse gas around the galactic nucleus \citep[e.g.]{Baan_1985}. There may be some evidence that this type of OH maser emission may be widespread, and only amplified to mega-maser levels in cases of extreme starbursts. Anomalous galactic OH maser emission has been detected within a parsec of warm-IR star-forming regions which match most of the properties (line ratios, unpolarized) seen in extragalactic mega-masers \cite{Mirabel_1989}.  The compact emission regions, in contrast, are thought to have high gains, leading to saturated maser emission \citep[e.g.,]{lonsdale2002}. Each compact source would be an accumulation of a few to many unresolved maser spots, each with narrow linewidths \citet{lo2005}.  \citet{Parra_2005} introduced a model capable of accounting for both types within a single gas phase, consistent with gas properties expected in these regions. \citet{randell1995} also show that the 1667 MHz is preferentially amplified for parsec scale clouds with an OH density of $10^4$ cm$^{-3}$ and warm dust ($40-50$ K), which are easily achievable within LIRGs \citep{lo2005}. The case for OH mega-masers not associated with LIRGs is less clear, though cases of these may be related to the systems hosting H$_2$O mega-masers (\S\ref{sec:oh_and_h2o}).

Along with the properties explained above, there are some more minor aspects or unique instances that should be mentioned. One detection in the survey of \citet{darling2002_paperIII} shows the first time variable OH mega-maser in IRAS 21272+2514 \citep[$z\sim 0.15$]{darling2002_timevar}. The variation shows maximum flux changes of 34\% the maximum observed flux. Variations are broadband, and the author's state that the modulation timescale is expected to fall within the the maximum timescale measured timescale, 821 days. Variations are also observed in H$_{2}$O megamasers (\S\ref{sub:h2o_props}). 

\begin{itemize}
\item typical linewidths 
\item multi-velocity emission components
\item trace highly obscured nuclear regions
\item time variability constraints - Darling paper IV
\item 10^5 times more luminous than galactic GMCs (Downs, Solomon 1998)
\item not necessarily amplification of bkg IR radiation
\item line profile wings explained as molecular outflows, starburst winds (Baan haschick 1987, baan haschick henkel 1989)
\end{itemize}

\subsection{Probing Nuclear Regions}
\label{sub:oh_highres}

High-resolution studies of several OH mega-masers have been conducted with the VLA, VLBI, and MERLIN. These provide a resolution of a few parsecs with the VLBI, and of order 100 pc with the VLA and MERLIN. These observations allow for the mega-maser sources to be connected with other high-resolution observations taken of the radio continuum and molecular gas (e.g., using CO(2-1) as a tracer). In both Arp 220 and Arp 299, both on-going merger systems, the OH mega-maser emission originates from within 100 pc of the nucleus, where the most intense star formation is occurring \citet{Lonsdale_1994, Baan_1990}. In the case of Arp 299, \citet{Baan_1990} show that the emission originates from a 100 pc rotating structure about the nucleus of IC 694.

\begin{itemize}
\item 
\item not necessarily amplification of bkg IR radiation
\end{itemize}

\subsection{Galaxy Mergers}
\label{sub:oh_mergers}

The evolution of galaxies remains a substantial question in astrophysics. Observationally, it has long been established that major galaxies will undergo interactions and mergers, and many systems have been observed in this process. The massive tidal disruptions give rise to near global instability in the molecular (and maybe atomic) ISM, leading to extreme outbursts. The clear relation between LIRGs and ULIRGs to major mergers, and the ability for OH mega-masers to trace regions near to the nucleus, makes them excellent tools for understanding the evolution of galaxies through mergers. However, of the observed ULIRGs, only $\sim 1/3$ exhibit OH mega-maser emission \citep{darling2002_paperIII}. This drops to $1/5.5$ in the entire sample including LIRGs.



This raises the obvious question put forward by \citet{lo2005}: what properties of these galaxy mergers give rise of OH mega-maser emission? What does this tell us about the merger process? Significant work has been accomplished in this area within the last few years, and many clear answers to these questions have arisen \citep{darling2012}. The correlation with FIR luminosity is well-established... 


The clear relation between OH mega-masers and major galaxy mergers leads to another potential applications, such as understanding the formation of super-massive black holes (SMBHs) and constraining cosmological parameters (this is already being done for H$_2$O mega-masers, \S\ref{sub:h2o_cosmo}). 

\begin{itemize}
\item LIRGS w/ and w/o mm emission (Lo); BGC 6240 examples if ULIRG w/o mm emission (Tacconi 1999)
\item role of IR radiation
\item Arecibo survey (Darling papers)
\item properties + relations determined in Darling paper III
\item relation to H2CO emission - all non-masing merges show 6cm H2CO absorption, all masing show emission (Mangum+08)
\item ~100 pc extents from intense starbursts w/ high concentrations of molecular gas
\item relation to CO luminosity (Darling+07) - OHM hosts break-away from the normal IR-CO correlation in SF galaxies (Gao solomon, 04)
\item seemingly unrelated to OH abundance  (Darling+12)
\item relation to IR spectra (Willett+2011a,b) - deep 10 um silicate absorption, steeper 20-30 um dust cont slope
\item limited time w/ ideal conditions during the merger?? - Darling+12
\item dust temp and optical depth significantly different b/w mm merger galaxies and those w/o - (Willett+11b)
\end{itemize}

FIGURE 4 from Darling+ 02

\begin{itemize}
\item for 1667 from Arecibo OHM survey
\item consistent slope with function found for ULIRGs (Kim and Saunders 98)
\item study rate of SMBH mergers, and galaxy mergers in general -- constraints on grav wave bkg??
\item powerlaw form
\item corrections for Malmquist bias, etc...
\item will be detectable to z~4 w/ full SKA
\item GMRT could be capable of detecting dozens more
\item set constraints on galaxy formation during most active epochs (ie. z~2)
\item independent constraint on merger rates; accomanying those from optical w/ HST (Kim, Saunders 98)
\item extinction free redshift determination from sub-mm galaxies
\end{itemize}

\subsection{Magnetic fields in Extreme Starbursts}
\label{sub:oh_zeeman}

The star formation process is poorly understood due to the many dynamical components, over a large range of scales, that may play vital roles. One important constraint to understand how star formation occurs in a particular environment is the magnetic field XXX Ostriker \& McKee 07 XXX. 

\begin{itemize}
\item McBride+14, mcbride heiles 13, robishaw heiles 09, robishaw+08
\item robishaw+08 measure splitting at 1667 in 5 ULIRGS w/ Arecibo+GBT (8 in survey)
\item B_LOS ~ .5 - 18 mG -- similar to galactic ones, suggesting local massive star formation is similar in starbursts
\item similar to B measured from synch radiation -- fields pervade multiple ISM phases
\item Equation 2 in robishaw+08 for B_LOS -- adapted Equation 1 in mcbride heiles 13
\item Some profiles show multiple components -- total of 14 individual components (spatial)
\item 1667 dominates over 1665, despite measuring similar B values to galactic -- maybe due to wider extragalactic maser lines
\item in Arp 220 - 4 spots - 2 w/ fields toward us and 2 w/ fields away
\item VLBI follow-up would give high-resolution field maps of B about the nucleus - robishaw heiles 09
\item additional 11 confident detections - mcbride heiles 13, 6.1-27.6 mG
\item new detections suggest larger than typical galactic B-fields -- median of 12 mG, ~2 larger than the Fish et al 2005
\item used all confirmed targets from Darling paper III and Willett 2012
\item Some non-detections may be due to much larger B, where the narrow line width approx doesn't hold
\item Can constrain some dynamics -- spherical cloud stability inferred suggests B plays a significant role in region dynamics, lws of ~20 km/s from Parra et al 05 in III Zw35 suggest non-gravitationally bound clouds. Magnetic support may keep confinement (not unlike some proposed schemes for cloud substructure)
\item McBride+14 include other B estimates from other methods, determine that ISM B-fields are greater in ULIRGs
\item dependent on minimum energy argument holding in the center of starburst galaxies
\item u_B > u_photon -- synchrotron cooling dominates over inverse Compton cooling -- CR electrons radiate energy via synchroton before leaving galaxy
\end{itemize}
