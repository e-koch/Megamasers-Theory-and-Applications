\section{OH Megamasers}
\label{sec:OH}

Hydroxyl (OH) produces maser emission in its ground state (see Table \ref{subsub:rad_pump}). Two levels of splitting occur in this ground state: (1) the unclosed shell of electrons in the ground state gives rise to $\Lambda$-doubling of all rotational levels, and (2) hyperfine splitting occurs between the unpaired electron and the H atom in the nucleus. Thus, there are four transitions in this ground-state with frequencies of 1612, 1665, 1667, and 1712 MHz with LTE line strengths of 1:5:9:1 \cite{lo2005}. The inversion mechanism accounting for all four transitions is complicated \citep{Elitzur_1992} and the relative maser line strengths can change drastically depending on the source's environment \citep{lo2005}. Masering in all four levels is extremely rare, and OH mega-maser detections are most often seen in the main 1665 \& 1667 MHz transitions, with 1667 MHz emission often dominating for mega-masers. This is the opposite situation of galactic masers associated with HII regions, where the 1665 MHz line is the dominant feature in the spectrum. Supernovae remnants show significant emission in the 1712 MHz line. These differences point to differences in the environmental properties where the maser occurs. 

There are approximately 100 detected mega-masers to date (some detections have been called into question or cannot be adequately determined in the literature, see \citet{darling2002_paperIII}). The majority of these detections have come from large-scale surveys with explicit source selection. The most successful and complete of these surveys to date is that by \citet{darling_2000_paperI} with the Arecibo telescope, which resulted in 52 new OH mega-maser detections to $z<0.23$ \citep{darling2002_paperIII}. This survey, combined with previous detections over the previous few decades, have allowed the basic properties of the OH mega-masers and host galaxies to be fairly-well understood. These processes trace the environment within $\approx 100$ pc of the galactic nuclei in regions of extreme starbursts. These regions are typically highly obscured, making it difficult to study their nature at other wavelengths.  Current instrumentation is capable of detecting galactic strength masers out to $\approx 20$ Mpc \citep{darling2012}, potentially allowing for the discovery of many more OH mega-masers in the near future. Such studies could constrain many important properties of starbursts and galaxy mergers to significant redshifts.


\subsection{Extragalactic Properties}
\label{sub:oh_gal_props}

The dominance of 1665 MHz line emission over the 1667 MHz emission from OH mega-masers suggests they form in a different environment than OH galactic masers. Current detections support this, as nearly all OH mega-masers occur in luminous or ultra-luminous infrared galaxies (LIRGs/ULIRGs). These galaxies are unique since they harbour some of the most extreme starburst activity in the universe XXX CITE XXX, and are almost exclusively caused by major galaxy mergers \citep{clements1996}. In fact, LIRGs and ULIRGs are classified based on their far infrared (FIR) luminosities alone ($L_{\mathrm{FIR}} \ge 10^{11.4}$ L$_{\odot}$ for LIRGs, $L_{\mathrm{FIR}} \ge 10^{12}$ L$_{\odot}$).  The star-formation activity is centered around at least one of the nuclei in the system. OH mega-masers tend to be found within $\sim 100$ pc of this nucleus, where there is an abundance of heated dense molecular gas \citep{lo2005}. The extreme star formation rates near the nucleus lead to widespread heating of the dust and gas in the ISM through many giant HII regions, resulting in elevated emission in the IR. \citet{Elitzur_1992} note that IR radiation is the only pumping agent capable of permeating the large galactic volumes required. This connection of a dense, heated molecular gas environment to OH maser emission was recognized early on by \citet{Bottinelli_1987}, as such conditions are the observed locations of galactic OH maser emission. 

The relation between the far-infrared (FIR) luminosity and the isotropic OH maser luminosity was rigorously established using 95 detections by \citet{darling2002_paperIII}. They find a power-law relation of $L_{\mathrm{OH}} \propto L_{\mathrm{FIR}}^{1.2\pm0.1}$, after thoroughly correcting for the biases in their sample. This updated the initial relationship found by \cite{Baan_1989}, which had an index of 2.  This index fits within the expectation for a mix of saturated and unsaturated masers. \citet{darling2002_paperIII} show that high-gain, unsaturated masers are expected to have an index 2, while an index of 1 should result for low-gain saturated masers. This is related to the source of the stimulated emission: the unsaturated case stimulates the background radio continuum, which itself is correlated with the FIR radiation \citep{Yun_2001}. Furthermore, this result establishes that, despite differing from galactic OH maser emission, OH mega-masers arise due to radiative pumping. This relationship suggests that if even brighter galaxies than the known ULIRGs are discovered, one may find a class of {\it giga}-masers, particularly if mega-maser detections out to $z\sim 2$ can be detected.  Two such systems were detected by \citet{darling2002_paperIII}.  The properties of OH mega-maser hosts are more thoroughly explored in \S\ref{sub:oh_mergers}.

There are two observed components of OH mega-masers, diffuse emission on $\sim 100$ pc scales, and parsec scale compact regions. Both of these components appear to only require fairly ordinary conditions \citep{lo2005}. The diffuse component occurs from low-gain, unsaturated masers which amplify continuum emission from the heated, diffuse gas around the galactic nucleus \citep[e.g.]{Baan_1985}. There may be some evidence that this type of OH maser emission may be widespread, and only amplified to mega-maser levels in cases of extreme starbursts. Anomalous galactic OH maser emission has been detected within a parsec of warm-IR star-forming regions which match most of the properties (line ratios, unpolarized) seen in extragalactic mega-masers \cite{Mirabel_1989}.  The compact emission regions, in contrast, are thought to have high gains, leading to saturated maser emission \citep[e.g.,]{lonsdale2002}. Each compact source would be an accumulation of a few to many unresolved maser spots, each with narrow linewidths \citet{lo2005}.  \citet{Parra_2005} introduced a model capable of accounting for both types within a single gas phase, consistent with gas properties expected in these regions. \citet{randell1995} also show that the 1667 MHz is preferentially amplified for parsec scale clouds with an OH density of $10^4$ cm$^{-3}$ and warm dust ($40-50$ K), which are easily achievable within LIRGs \citep{lo2005}. The case for OH mega-masers not associated with LIRGs is less clear, though cases of these may be related to the systems hosting H$_2$O mega-masers (\S\ref{sec:oh_and_h2o}).

% Along with the properties explained above, there are some more minor aspects or unique instances that should be mentioned.
One detection in the survey of \citet{darling2002_paperIII} shows the first time variable OH mega-maser in IRAS 21272+2514 \citep[$z\sim 0.15$]{darling2002_timevar}. The variation shows maximum flux changes of 34\% the maximum observed flux. Variations are broadband, and the author's state that the modulation timescale is expected to fall within the the maximum timescale measured timescale, 821 days. Variations are also observed in H$_{2}$O megamasers (\S\ref{sub:h2o_outflows}). \citet{darling2002_timevar} favour interstellar scintillation as the cause of the variability, however their is claim is tentative and will likely require discovering more sources with this behaviour. Assuming the scintillation model, they constrain the size of the variable source to be smaller than 2 pc, based on the modulation timescales. This is consistent with high-resolution observations of compact masing regions (\S\ref{sub:oh_highres}). XXX Lonsdale et al variability in Arp 220 XXX

In some of the first detections of OH mega-masers, the line profiles were observed to exhibit extended `wings' in certain velocity features \citep{Baan_1987,Baan_1989}. \citet{Baan_1989} combined the OH maser observations with HI for three systems (III Zw 35, IR 12112+0305, and IC 4553) to confirm the presence of molecular outflows. These outflows are consistent with the velocities of starburst-driven superwinds found in some other active galaxies. 

\begin{itemize}
\item typical linewidths 
\item multi-velocity emission components
\end{itemize}

% \subsection{Probing Nuclear Regions}
% \label{sub:oh_highres}

% High-resolution studies of several OH mega-masers have been conducted with the VLA, VLBI, and MERLIN. These provide a resolution of a few parsecs with the VLBI, and of order 100 pc with the VLA and MERLIN. These observations allow for the mega-maser sources to be connected with other high-resolution observations taken of the radio continuum and molecular gas (e.g., using CO(2-1) as a tracer). In both Arp 220 and Arp 299, both on-going merger systems, the OH mega-maser emission originates from within 100 pc of the nucleus, where the most intense star formation is occurring \citet{Lonsdale_1994,Baan_1990}. In the case of Arp 299, \citet{Baan_1990} show that the emission originates from a 100 pc rotating structure about the nucleus of IC 694.

% \begin{itemize}
% \item 10^5 times more luminous than galactic GMCs (Downs, Solomon 1998)
% \item not necessarily amplification of bkg IR radiation
% \end{itemize}

\subsection{Galaxy Mergers}
\label{sub:oh_mergers}

The evolution of galaxies remains a substantial question in astrophysics. Observationally, it has long been established that major galaxies will undergo interactions and mergers, and many systems have been observed in this process. The massive tidal disruptions give rise to near global instability in the molecular (and maybe atomic) ISM, leading to extreme outbursts. The clear relation between LIRGs and ULIRGs to major mergers, and the ability for OH mega-masers to trace regions near to the nucleus, makes them excellent tools for understanding the evolution of galaxies through mergers. However, of the observed ULIRGs, only $\sim 1/3$ exhibit OH mega-maser emission \citep{darling2002_paperIII}. This drops to $1/5.5$ in the entire sample including LIRGs.

This raises the obvious question put forward by \citet{lo2005}: what properties of these galaxy mergers give rise of OH mega-maser emission? And what does this tell us about the merger process? Significant work has been accomplished in this area within the last few years, and many clear answers to these questions have arisen \citep{darling2012}. The correlation with FIR luminosity is well-established \citep{darling2002_paperIII}, however there is substantial scatter in the correlation which hinders its use for distinguishing between masing and non-masing LIRGs. This is not suprising, since the scales over which each type of emission occurs is vastly different (FIR luminosity is measured over the entire galaxy). \citet{darling2002_paperIII} also explored the correlation of OH maser luminosity to other features of the host galaxy, notably the IR `colour' $\log_{10}(f_{100\mu\mathrm{m}}/f_{60\mu\mathrm{m}})$ (see Figure \ref{fig:oh_props}). \citet{darling2002_paperIII} utilize Kapler-Meier (K-M) survival analysis \citep{Feigelson_1985} to explore the significance between the masing and non-masing distributions in their survey. They find that there is a strong rise in FIR colour between $\sim0.4-0$ for OH mega-maser hosts, indicating that LIRGs with `warmer' colours may be preferred. Their K-M analysis confirms this to high confidence. \citet{darling2002_paperIII} also explore the further relations between the FIR and OH maser properties, without a clear result. 

\cite{Darling_2007} extended this search by exploring the connection between star formation rate (calculated from IR luminosity) and the line luminosity of CO. A linear relationship exists between the IR and CO luminosities for star-forming galaxies over many orders of magnitude \citep{Gao_2004}. \citet{Darling_2007} finds the OH mega-maser hosts break this linear trend XXX FIGURE XXX, suggesting that the hosts of OH mega-masers are undergoing a special triggered star formation event due to the merger.

Further differences between the OH mega-maser hosts and LIRGs without maser emission were found by \citet{willett2011_I,willett2011_II}. In these two papers, the authors use Spitzer IRS spectroscopy to study LIRGs and ULIRGs that were part of the \citet{darling2002_paperIII} survey, both detections and non-detections. The medianed spectra of all sources is shown in Figure \ref{fig:oh_IR_spectra}. There is a clear difference between the galaxies with and without OH mega-masers. Notably, hosts of OH mega-masers show deeper absorption features near 10 and 18 $\mu$m, and the slope at longer wavelengths is steeper. \citet{ivezic1997} show that these are features due to an increase in dust opacity. This also shifts the peak of the IR emission to a maximum between 35-53$\mu$m, which is likely the pumping mechanism for the OH masers \citep{darling2012}. \citet{willett2011_II} then derive the dust temperature and optical depths in their selection of sources. \citep{darling2012}. The model of \citet{lockett2008} predicts that OH mega-masing galaxies should fall in a locus on the dust temperature-optical depth plane, well-separated from non-masing galaxies. The observations of \citet{willett2011_I} do show this separation well XXX FIG?? 9 XXX, however the inferred optical depths for the OH maser hosts are an order of magnitude larger than the predicted values. This result is key for determining targets in future surveys \citep{darling2012}.

A further piece of evidence comes from the 6.3 cm H$_{2}$CO line. \citet{Mangum_2008} surveyed nearby star-forming galaxies for this line, and found a clear separation between non-masing galaxies and OH mega-maser hosts: all non-masing galaxies show the line in absorption, while maser hosts show it in emission. This line is collisionally driven to low excitation temperatures and flips between emission and absorption when the excitation temperature crosses the CMB temperature \citep{darling2012}. This suggests that OH mega-maser emission may have a density threshold.

XXX darling 2006?? XXX

These factors point to the cause OH mega-maser emission being a phase of the merger process where extreme concentrations of gas and dust are evident.

The clear relation between OH mega-masers and major galaxy mergers leads to another potential applications, such as studying super-massive black hole binaries, highly-obscured star formation, merger rates of galaxies \citep{darling_2001_merger} and constraining the gravitational wave background. These are the applications proposed by \citet{darlin2002_lumfunc}, where the authors perform an excruciatingly careful and thorough analysis to derive the OH mega-maser luminosity function:
\begin{equation}
\label{eq:oh_lum}
\Phi = \left( 9.8^{+31.9}_{-7.5} \right)L_{\mathrm{OH}}^{-0.64\pm0.21} \mathrm{Mpc}^{-3} \mathrm{dex}^{-1}
\end{equation}
where $L_{\mathrm{OH}}$ is in solar luminosities. This was derived using the samples in \citet{darling2002_paperIII}. The powerlaw index is consistent with that derived by \citet{Kim_1998} for ULIRGs ($-0.94\pm0.12$). This luminosity function forms the basis of studying galaxy evolution using OH mega-masers as tracers. In the case of merger rates, it will provide an independent estimate from optical studies conducted with the Hubble Space Telescope \citep{Kim_1998}. Such applications require far more significant population studies of OH mega-masers. Detection rates using next-generation radio telescopes are given in \S\ref{sec:conclusion}.

\subsection{Magnetic fields in Extreme Starbursts}
\label{sub:oh_zeeman}

The star formation process is poorly understood due to the many dynamical components, over a large range of scales, that may play vital roles. One important constraint to understand how star formation occurs in a particular environment is the magnetic field XXX Ostriker \& McKee 07 XXX. Since these regions of extreme star-formation are highly obscured (see above), OH mega-masers offer a unique look into the magnetic field properties through Zeeman splitting. OH is the only observed para-magnetic masing species \citep{Elitzur1992_text}, since the outer electron shell contains an unpaired electron. Galactic OH masers associated with HII regions are typically strongly circularly polarized, as is expected for a para-magnetic species with a fully resolved Zeeman pattern \citep{Elitzur1992_text}. Megamasers are not 

\begin{itemize}
\item McBride+14, mcbride heiles 13, robishaw heiles 09, robishaw+08
\item robishaw+08 measure splitting at 1667 in 5 ULIRGS w/ Arecibo+GBT (8 in survey)
\item B_LOS ~ .5 - 18 mG -- similar to galactic ones, suggesting local massive star formation is similar in starbursts
\item similar to B measured from synch radiation -- fields pervade multiple ISM phases
\item Equation 2 in robishaw+08 for B_LOS -- adapted Equation 1 in mcbride heiles 13
\item Some profiles show multiple components -- total of 14 individual components (spatial)
\item 1667 dominates over 1665, despite measuring similar B values to galactic -- maybe due to wider extragalactic maser lines
\item in Arp 220 - 4 spots - 2 w/ fields toward us and 2 w/ fields away
\item VLBI follow-up would give high-resolution field maps of B about the nucleus - robishaw heiles 09
\item additional 11 confident detections - mcbride heiles 13, 6.1-27.6 mG
\item new detections suggest larger than typical galactic B-fields -- median of 12 mG, ~2 larger than the Fish et al 2005
\item used all confirmed targets from Darling paper III and Willett 2012
\item Some non-detections may be due to much larger B, where the narrow line width approx doesn't hold
\item Can constrain some dynamics -- spherical cloud stability inferred suggests B plays a significant role in region dynamics, lws of ~20 km/s from Parra et al 05 in III Zw35 suggest non-gravitationally bound clouds. Magnetic support may keep confinement (not unlike some proposed schemes for cloud substructure)
\item McBride+14 include other B estimates from other methods, determine that ISM B-fields are greater in ULIRGs
\item dependent on minimum energy argument holding in the center of starburst galaxies
\item u_B > u_photon -- synchrotron cooling dominates over inverse Compton cooling -- CR electrons radiate energy via synchroton before leaving galaxy
\end{itemize}
