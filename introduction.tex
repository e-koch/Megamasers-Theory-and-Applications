\section{Introduction}
\label{sec:intro}

Since the discovery of cosmic masers by XXX Gundermann 1965 XXX, detections of amplified microwave emission have been found in a variety of astronomical environments from numerous chemical species. "Maser" is an acronym for microwave amplification by stimulated emission of radiation, a phenomenon first discovered in laboratory settings by XXX Gordon, Zeiger, Townes 1955 XXX, just a decade before the first astronomical detections.



\begin{itemize}
\item arise in conditions where the population is inverted (out of thermal equilib)
\item requires excitation source or energy source
\item found to occur in multiple astrophysical environments (galactic: late-type stars, HII regions, jet interaction, comets; extragalactic: circumnuclear disks)
\item mega-masers found in H$_2$O, OH, CH$_3$OH, SiO, H$_2$CO
\item observations of kilo-gigamasers (based on galactic maser luminosities)
\item 
\end{itemize}

This report is arranged as follows. Section \ref{sec:maser_theory} presents the basics of Maser Theory, deriving expressions for the line intensity as a function of path length in the unsaturated and saturated regimes. Section \ref{sec:OH} presents the observational results and applications of OH megamaser emission, while Sections \ref{sec:H2O} and \ref{sec:others} present the same for H$_2$O and other detected megamasers, respectively. Section \ref{sec:conclusion} discusses the use of future surveys using the upcoming generation of radio telescopes.