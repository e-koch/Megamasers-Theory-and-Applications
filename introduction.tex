\section{Introduction}
\label{sec:intro}

Since the discovery of cosmic masers by \citet{gundermann} and \citet{WEAVER_1965}, detections of amplified microwave emission have been found in a variety of astronomical environments from numerous chemical species. "Maser" is an acronym for microwave amplification by stimulated emission of radiation, a phenomenon first discovered in laboratory settings by \cite{Gordon_1955}, just a decade before the first astronomical detections. The conditions required for maser emission occur only in environments out of thermal equilibrium \citep[e.g.,]{Elitzur_1992}. Typical conditions in the interstellar medium (ISM) are capable of inducing such conditions, primarily due to low densities of material (with respect to terrestrial conditions). Thus in many ISM environments, collisional processes do not dominate \citep[e.g.,]{stahler_palla_2004}.

Due to their non-thermal origins, masers have extremely high brightness temperatures ranging from $T_{\mathrm{b}} = 10^{10}-10^{14}$ K \citep{lo2005}. Such high brightness temperatures obviously do not correspond to the physical temperatures of the emitting material. Instead, the brightness temperatures correspond to level populations that do not follow a Boltzmann distribution \citep{Elitzur_1992}. In the Rayleigh-Jeans limit, brightness temperature has the form: 
\begin{equation}
\label{eq:temp_bright}
T_{\mathrm{b}} = \frac{S_\nu}{\Omega_{\mathrm{s}}} \frac{\lambda^2}{2k},
\end{equation}
These elevated brightness temperatures arise due to the small solid angle of the emission regions ($\Omega_{\mathrm{s}}$), and large surface brightnesses ($S_\nu$).

Maser emission has been detected in various astrophysical environments, such as around late-type stars, embedded protostars, HII regions, interactions between jets and the surrounding ISM, and comets \citep{lo2005}. Typically, these emission lines can be classified as Type I masers, driven by collisional processes such as stellar winds or jets hitting dense gas, or Type II masers, due to radiative excitement from a nearby star or energetic source. These represent the source which provides the energy to invert the population levels of the masing species. In maser theory, these energy sources are the "pumping mechanisms", which may be collisionally (Type I) or radiatively (Type II) driven (see \S\ref{sec:maser_theory}).

In this report, I focus on the discovery, mechanisms, and applications of extragalactic mega-masers. It should be noted that no galactic mega-masers have been detected, and it is unlikely they occur within non-AGN or starburst galaxies. Until recently, mega-maser emission had only been confidently detected from three species: OH, H$_2$O, and H$_2$CO. The vast majority of the detected mega-maser detection arises from the 22 GHz $6_{16} \longarrow 5_{23}$ H$_2$O transition and the primary hyperfine ground-state OH lines (1665 \& 1667 MHZ). Mega-masers were first detected serendipitously by \citet{DOS_SANTOS_1979}, who found 22 GHz H$_2$O emission towards the nucleus of NGC 4945. They inferred an isotropic luminosity of $\sim 85$ L$_{\odot}$, $~10^6$ times greater than any known galactic maser source, leading to the term "mega"-maser. \citet{Baan_1982} reported the first OH mega-maser towards the peculiar galaxy IC 4553 with an inferred luminosity of $\sim10^3$ L$_{\odot}$. \citet{baan1986} reported the lone detection of a formaldehyde (H$_2$CO) mega-maser in the 1$_{10}$-1$_{11}$ (6.2 cm) transition. Recent results by, \citet{wang2014} and \citet{chen2015} have extended the mega-masing species to five with their detection of six transitions in methanol (CH$_3$OH; 36.2 GHz, 37.7 GHz, and four clustered at 96.7 GHz), and the silicon monoxide (SiO) J=2-1  ($v=3$) transition. It should be noted that there are detections of {\it kilo}-masers towards the nucleus of nearby galaxies \citep[e.g.,]{Ho_1987}, but it is unclear if these arise from the same processes of as mega-maser \citep{lo2005}. Several properties of the observed mega-masers do not match those seen from the same species in galactic sources. This strongly suggests that mega-masers are powered by different energy sources.

Observervations of mega-masers show that most arise from emission regions on the order of milli-arcseconds in diameter. Coupled with their high surface-brightness, they make excellent VLBI targets. Most VLBI observations of mega-masers show that the majority of the flux is recovered using the VLBI alone, confirming that there is little mega-maser emission on scales larger than these compact source \citep{lo2005}.

This report is arranged as follows. Section \ref{sec:maser_theory} presents the basics of Maser Theory, solutions to the radiative transfer... Section \ref{sec:OH} presents the observational results and applications of OH mega-maser emission, while Sections \ref{sec:H2O} and \ref{sec:others} present the same for H$_2$O and other detected mega-masers, respectively. Section \ref{sec:conclusion} discusses the use of future surveys using the upcoming generation of radio telescopes.
