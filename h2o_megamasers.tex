\section{H_2O Megamasers}
\label{sec:h2o_mm}

Multiple H$_2$O transitions have been observed producing maser emission in galactic source \citep{Elitzur_1992}, however the dominant transition detected as stimulated emission is the 22 GHz transition (see Table \ref{tab:maser_props}). This transition is relatively easy to cause stimulated emission in since it corresponds at a `backbone' transition (lowest levels at each rotational $J$-value). These transitions have large Einstein $A$-values, leading to trapping of the emitted photons \citep{stahler_palla_2004}. This makes collisional excitement and de-excitement more effective than radiation in producing these transitions. The environment needed to produce the 22 GHz H$_2$O emission must pump molecules to fairly high energy levels (640 K, Table \ref{tab:maser_props}). As shown by \citet{stahler_palla_2004}, shocked regions are capable of providing collisional excitement to these levels, and indeed this is where H$_2$O maser emission is observed in both galactic and extragalactic source \citep{Elitzur_1992, lo2005}. 

The 22 GHz H$_2$O maser emission is associated with circumstellar material around late-type stars, and with molecular outflows due to young, embedded stars. These environments suggest a collisional pumping mechanism. H$_2$O mega-maser emitting regions are typically found within a few parsecs of an AGN, coinciding with regions of dense molecular gas heated to suitable temperatures for masers to occur. 

\begin{itemize}
\item bennert+09, tarchi+12, tarchi+07, cernicharo+06, baan+06, baan klockner 05, baan+82, dos santos lepine 79
\item 
\end{itemize}

\subsection{Extragalactic versus Galactic Properties}
\label{sub:h2o_props}

Galactic

\begin{itemize}
\item 
\end{itemize}

Extragalactic

\begin{itemize}
\item sales+15, wagner+13, raluy+1998
\item circumnuclear and nuclear jet driven masers and outflow
\item Detection rates: 150/3000 searched galaxies show H2O MM (tarchi+2012)
\item majority are Seyfert 2 or LINERs (z<0.05)
\item much higher detection rates simply among these 2 classes (~25\%, tarchi+12)
\item AGN w/ H2O MM show high N_H > 10^23 cm^-2, some are compton-thick (>10^24)
\item rough correlation w/ unabsorbed X-ray luminosities (kondratko+06) -- uncertain due to poor X-ray data for 1/2 of MM detections
\item Properties of AGN w/ MM: primarily in X-ray absorbed source (zhang+10, ramolla+11, zhu+11); have larger radio luminosities (zhang+12), sample based on extinction-corrected [OIII]5007 flux (zhu+11)
\item highest occurence in Seyfert 2 galaxies -- not suprising since Unified model as it predicts an edge-on disk or torus (ie giving longest path length), tarchi+12
\item Some Seyfert 1 detections have occurred -- more unclear since Unified model predicts these should be face-on
\item Only 1/150 searches of pure Sy1 have given detection, but seem to occure more often with narrow line Sy1's (narrowst balmer lines, strongest FeII emission, exteme X-ray properties (komossa 08))
\item 5/71 detections in this subclass (tarchi+11b) (~7\% rate), sample is volume-limited (those within 1e4 km/s boost rate to ~24\%, highest rate of any AGN class)
\item many proposed reasons: intermediate viewing angles, accretion rates near eddington, strong outflows (tarchi+12)
\item No detections in elliptical or radio-loud galaxies -- several surveys conducted (see refs in Tarchi+12); seems regardless of the AGN type
\item however, there are 4 unclassified H2O MM galaxies which show unique properties (5 including NGC 1052)
\item Could be due to: lack of molecular gas (henkel+98); instability from tidal disruption in clouds orbiting particularly massive SMBHs (tarchi+07b); insufficient sensitivity of completed surveys
\end{itemize}

\subsection{AGN: Disks \& SMBH masses}
\label{sub:h20_smbh_mass}

\begin{itemize}
\item greene+2013, Wardle Yusef-zadeh 12, greene+10, schulz henkel 03, tarchi+12
\item these are circumnuclear masers
\item Unified model of AGN (antonucci 93, urry+padovani 95) predicts an accretion disk about AGN surrounded by torus or thick disk
\item Type 1 AGN - view accretion disk and BH through hole in torus; type 2 - direct view of nucleus obscured by torus
\item masers are unique tool to probe ~pc regions around AGN -- other bands are highly obscured
\item characteristically have triple peak in velocity -- center at systematic vel of galaxy, w shifted by hundreds of km/s -- rotation of disk
\item NGC 4258 best studied for this  (Miyoshi+95, Herrnstein+99) -- MCP has greatly extended this
\item 
\end{itemize}

\subsection{Jet masers}
\label{sub:h2o_jets}

\begin{itemize}
\item tarchi+12
\item interaction b/w radio jet and MCs along path; or amplification of bkg radio continuum from jet through foreground cloud
\item display single broad emission feature -- two examples are Mrk328 NGC1052 (Falcke+2000, braatz+94)
\item maser located along radio continuum of jet, separate from nucleus pos'n in both cases (pek+03 1052, claussen+98 328)
\item peck+03 used "reverberation mapping techniques" to constrain jet properties (density and velocity) -- maser from shock region b/w jet and MC
\item in 1052, sawada-satoh+08 find maser occurring in foreground of jet -- cloud amplifying radio jet emission in circumnuclear torus
\item maser where free-free opacity of thermal plasma absorbing synch rad is large
\item contraction of cloud towards nucleus gives redshifted features
\item 
\end{itemize}

\subsection{Outflow Masers}
\label{sub:h2o_outflows}

\begin{itemize}
\item tarchi+12, greeenhill+03, mccallum+05
\item Only detected in Circinus -- seyfert 2 
\item 0.2 pc radius disk where masers located (tristram+07), also a thick torus out to pc scales, slightly cooler than disk, show a clumpy distribution in torus
\item evidence for collimated AGN outflow w/ maser position
\item mccallum+03 confirm scintillation of H2O mm, rapid variability linked to interstellar scintillation seen in many flat spectrum AGN
\item orginal variation obersved by greenhill+97
\item can test for annual cycles in scintillation, constraining the peculiar velocity of the scattering materialm estimate anisotropies in the medium -- create model of the source
\item can also lead to direct measurement of ISM velocity -- important for constraining any model of anisotropic structure
\item weak scintillation model explains observed variability assuming a local screen; distant, strongly scattering screen model fits observations, shows unseen anti-correlation b/w modulation index and timescale
\end{itemize}

\subsection{Cosmology: Distance Determination}
\label{sub:h2o_cosmo}

\begin{itemize}
\item Megamaser comology project papers
\item 
\end{itemize}

\subsection{Connection between OH and H$_2$O?}
\label{sec:oh_and_h2o}

\begin{itemize}
\item wiggins+15, tarchi+12
\item OH megamaser seem to be uniquely associated with LIRGs and ULIRGs (L_IR>10^11 L_sol) -- ideal for IR radiative pumping 
\item Darling Paper III IR OH correlation
\item 
\end{itemize}