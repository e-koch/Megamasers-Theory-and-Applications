\section{H_2O Megamasers}
\label{sec:h2o_mm}

Multiple H$_2$O transitions have been observed producing maser emission in galactic source \citep{Elitzur_1992}, however the dominant transition detected as stimulated emission is the 22 GHz transition (see Table \ref{tab:maser_props}). This transition is relatively easy to cause stimulated emission in since it corresponds at a `backbone' transition (lowest levels at each rotational $J$-value). These transitions have large Einstein $A$-values, leading to trapping of the emitted photons \citep{stahler_palla_2004}. This makes collisional excitement and de-excitement more effective than radiation in producing these transitions. The environment needed to produce the 22 GHz H$_2$O emission must pump molecules to fairly high energy levels (640 K, Table \ref{tab:maser_props}). As shown by \citet{stahler_palla_2004}, shocked regions are capable of providing collisional excitement to these levels, and indeed this is where H$_2$O maser emission is observed in both galactic and extragalactic source \citep{Elitzur_1992, lo2005}. 

The 22 GHz H$_2$O maser emission is associated with circumstellar material around late-type stars, and with molecular outflows due to young, embedded stars. These environments suggest a collisional pumping mechanism. H$_2$O mega-maser emitting regions are typically found within a few parsecs of an AGN, coinciding with regions of dense molecular gas heated to suitable temperatures for masers to occur. 

\begin{itemize}
\item bennert+09, tarchi+12, tarchi+07, cernicharo+06, baan+06, baan klockner 05, baan+82, dos santos lepine 79
\end{itemize}

\subsection{Extragalactic versus Galactic Properties}
\label{sub:h2o_props}

\begin{itemize}
\item sales+15, wagner+13, mccallum+05, raluy+1998
\item circumnuclear and nuclear jet driven masers
\end{itemize}

\subsection{AGN: Disks \& SMBH masses}
\label{sub:h20_smbh_mass}

\begin{itemize}
\item greene+2013, Wardle Yusef-zadeh 12, greene+10, schulz henkel 03
\end{itemize}

\subsection{Geometric distance measurements}
\label{sub:h2o_distance}



\subsection{Cosmology}
\label{sub:h2o_cosmo}

\begin{itemize}
\item Megamaser comology project papers
\item 
\end{itemize}

\subsection{Connection between OH and H$_2$O?}
\label{sec:oh_and_h2o}

\begin{itemize}
\item wiggins+15
\end{itemize}