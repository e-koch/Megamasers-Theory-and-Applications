\section{H_2O Megamasers}
\label{sec:h2o_mm}

Multiple H$_2$O transitions have been observed producing maser emission in galactic source \citep{Elitzur_1992}, however the dominant transition detected as stimulated emission is at 22 GHz (see Table \ref{tab:maser_props}). This transition is relatively easy to cause stimulated emission in since it corresponds at a `backbone' transition (lowest levels at each rotational $J$-value). These transitions have large Einstein $A$-values, leading to trapping of the emitted photons \citep{stahler_palla_2004}, making collisional excitement and de-excitement more effective than radiation in producing these transitions. The environment needed to produce the 22 GHz H$_2$O emission must pump molecules to fairly high energy levels (640 K, Table \ref{tab:maser_props}). As shown by \citet{stahler_palla_2004}, shocked regions are capable of providing collisional excitement to these levels, and indeed this is where H$_2$O maser emission is observed in galactic, and possibly extragalactic, sources \citep{Elitzur_1992,lo2005}. 

Approximately 3000 galaxies have been searched for H$_2$O mega-masers, resulting in about 150 detections\footnote{based on information compiled from the Megamaser Cosmology Project, \S\ref{sub:h2o_cosmo}}  \citep{tarchi2012}. Nearly all of the detections come from active galaxies, specifically radio-quiet AGN classified as Seyfert 2 (Sy2) or low-ionization nuclear emission-line regions (LINERs) with redshifts up to 0.05 (\S\ref{sub:h20_agn}). These galaxies tend to all host the `circumnuclear' H$_2$O mega-masers associated with compact disks within parsec scales of the AGN \citep{lo2005}. This makes H$_2$O mega-masers a unique tool to study the properties of the AGN on small scales that are otherwise heavily obscured. Two other types of H$_2$O mega-masers are observed corresponding to interactions with a nuclear jet and molecular outflows \citep{lo2005, tarchi2012}. There are fewer detections of masers from these mechanisms, and so I focus on circumnuclear masers in this section. In galactic sources, 22 GHz H$_2$O masers are associated with circumstellar material around late-type stars, and with molecular outflows due to young, embedded stars. These environments, and the reasons stated above, highly suggest a collisional pumping mechanism \citep{Elitzur_1992}. Circumnuclear H$_2$O mega-masers appear to arise from a related mechanism to the circumstellar sources, except that the ultimate energy source of the pump is the AGN itself \citep{lo2005}. The energy output of the AGN is significantly greater than that from late-type stars, which possibly explains why these maser sources are much brighter. 

Numerous surveys have been undertaken to find H$_2$O mega-maser detections, notably by \citet{Braatz_1996}, who surveyed {\it all} nearby galaxies with AGN characteristics (see also \citet{Braatz_1997}). The Megamaser Cosmology Project is an on-going H$_2$O mega-maser survey of known sources that combines high- and low-resolution observations over long time periods to obtain geometric distances to galaxies beyond $\gt 100$ Mpc, and infer H$_0$ to a high precision, independent of other methods (\S\ref{sub:h2o_cosmo}) \citep{reid2009_mmproject_I}. Given the applications discussed in the preceding sections, many more H$_2$O mega-maser will be conducted as more sensitive instrumentation becomes available.

\subsection{Host galaxies}
\label{sub:h2o_hosts}

The detection rate for nearby Sy2 and LINER galaxies is about $1/4$, while the overall detection rate for all AGN observed is just $5\%$ \citep{tarchi2012}. What is the cause of this significant difference between different classes of AGN?

The Unified Model of AGN predicts the presence of the parsec-scale accretion disks about the SMBH surrounded by a torus or thick disk of atomic or molecular gas extending out to $\sim 100$ pc \citep[e.g.,]{Urry_1995}. The model broadly classifies AGN based on whether the observer accretion disk and SMBH are unobscured (Type I) or obscured by the torus (Type II). The differences in the specific radio classes of AGN are more relatable to the properties of the galaxy and AGN \citep{tarchi2012}. Sy2 galaxies are predicted to have an edge-on disk or torus from the observer's perspective. It then makes sense that most H$_2$O mega-maser detections come from this class since this geometric configuration maximizes the path length for the maser emission to arise from \citep{tarchi2012}.  Sy2 galaxies with very-high column densities ($\gt10^{24}$ cm$^{-3}$, inferred in the X-ray) seem to preferentially be hosts to H$_2$O mega-masers \citep{Braatz_1997,Greenhill_2003}. \citet{Kondratko_2006} also find a rough correlation between H$_2$O mega-maser luminosities and the 2-10 keV X-ray luminosity, though the X-ray data is poor for nearly half their survey objects \citet{tarchi2012}. Other indicators include the maser emission residing within X-ray absorbed sources \citep[e.g.,]{Zhang_2010}, and higher radio luminosities with respect to non-maser galaxies \citep{Zhang_2012}.

Seyfret 1 galaxies are predicted to be observed face-on; one H$_2$O mega-maser has been observed from this class in NGC 2782 and it remains unclear on how this arises \citep{Tarchi_2011}. However, \citet{Tarchi_2011} also find a high occurence rate (corrected for the volume-limited sample) for H$_2$O mega-masers in narrow-line Sy1 galaxies, a subset of Sy1 galaxies that exhibit narrow Balmer lines and strong Fe II emission \citep{tarchi2012}. This certainly complicates being able to predict the AGN properties mostly likely to result in H$_2$O mega-maser emission. Proposed explanations for this occurence include intermediate viewing angle between Sy1 and Sy2, accretion rates within the circumnuclear disk near the Eddington limit, small SMBH masses, and strong molecular outflows fueled by extreme radiation pressure \citep{tarchi2012}. Despite searches in $\sim 400$ galaxies, no H$_2$O mega-masers have been detected in radio-loud or elliptical galaxies, regardless of AGN type \citep{tarchi2012}. 


\subsection{Circumnuclear H$_2$O mega-masers \& Evidence for SMBHs}
\label{sub:h20_agn}

Circumnuclear H$_2$O mega-masers exhibit emission at three peaks in the velocity dimension: one centered at the systematic velocity of the AGN, and two red- and blue-shifted from the center by hundreds of km/s. Through high-resolution observations with the VLBI, the maser sources are resolved out to show that these components are distributed in a near planar distribution, where each velocity component arises from different regions in a parsec-scale disk around the nucleus. Assuming the disk is seen edge-on (as expected for Sy2 galaxies), the shifted velocity components correspond to components tangent to the line-of-sight on either side. This is direct proof of the expected circumnuclear disks predicted by the Unified Model \citep{lo2005}. An example observation is shown in Figure \ref{fig:h2o_ngc4258} for NGC 4258 \citep{Bragg_2000}, the most studied object in this class (see also \citet{Miyoshi_1994} \& \citet{Herrnstein_1999}). Other H$_2$O mega-maser spectra show similar properties, along with the further evidence for a rotating disk from the observed linear increase in the systematic velocity component \citep{lo2005}.  

Modelling these disk based on the maser components reveals Keplerian or sub-Keplerian motion, and the planar geometry of the resolved maser spots suggest a thin disk \citep{lo2005}. In the rotating disk model, the aforementioned constant rate of increase of the systematic velocity results from the centripetal acceleration $dv/dt = v_r^2/r$, where the radial velocities correspond to the high-velocity components. Then the constant velocity gradient can be written as
\begin{equation}
\label{eq:disk_rot}
\frac{dV}{d\theta} = D \frac{dV}{db} = D \frac{v_r}{R}
\end{equation}
where the impact parameter $b= \theta D$, with $\theta$ being the angular offset from the center of the disk \citep{lo2005}. Since $dV/dt$, $dV/d\theta$ can be measured, one can solve for the distance to the source. This was accomplished for the first time by \citet{Miyoshi_1994}, and is now being used in other systems as a means of measuring the Hubble Constant (\S\ref{sub:h2o_cosmo}). For NGC 4258, \citet{Herrnstein_1999} find a distance of $7.2\pm0.5$ Mpc, which remains one of the most accurate measurements for a galaxy.

The modeled rotation curves allow for the masses of the nuclei to be measured. While it is generally assumed that galactic nuclei host SMBHs, direct evidence is difficult to establish \citep{lo2005}. The proximity of H$_2$O mega-masers to the nuclei make them excellent tools to test for the existence of SMBHs. The mass to radius ratio should be $6.7\times10^{27}$ g/cm, using the Schwarzchild radius \citep{lo2005}. For NGC 4258, the Keplerian rotation curve yields a ratio of $2\times10^{25}$ g/cm from VLBI observations. Such a high-ratio is difficult to explain in terms of a supremely dense cluster of compact objects. Using the Keplerian rotation curves observed from seven other systems (NGC 6264, NGC 1194, NGC 2273, NGC 2960, NGC 4388, NGC 6323), \citet{kuo2011_mmproject_III} rules out a cluster of compact objects as potential explanation for galactic nuclei through explicit modeling based on Plummer profiles. This evidence is highly-compelling for the existence of SMBHs. \citet{kuo2011_mmproject_III} then assume that the nuclei are SMBHs, and combine previously published values for distance with the results of the Keplerian rotation curve model to infer black hole masses ranging between $0.76-6.5\times10^7$ M$_{\odot}$. The error on these measurements is about 11\%, with the dominant error source coming from uncertainties in the Hubble Constant\footnote{A value of $H_0=73\pm8$ km/s/Mpc was used \citep{Freedman_2001}.}.   

\citet{wardle2012_bhmass} have derived a model for the formation of these circumnuclear disks around AGN. They argue that impacts of molecular clouds, within some impact parameter, can cause the formation of a thin Keplerian disk. By equating the specific angular momentum at the edge of the disk to that of the material that can just barely be captured by the interaction, they find a relation between the radius of the circumnuclear disk and the mass of the SMBH: $R=4G\lambda^2M/v^2$, where $\lambda$ is the fraction of angular momentum maintained in the interaction, $M$ is the SMBH mass, and $v$ is the initial velocity of the cloud. They find that the relation holds for eight disks with H$_2$O mega-maser emission, largely within observational error. They suggest that the impact of a cloud on one of these disks could create gravitationally unstable episodes, which may potentially give rise to mega-maser emission \citep{Milosavljevi_2004} such that only a fraction host Keplerian circumnuclear disks.

{\bf XXX Constraining low mass end of M-sigma relation XXX}

% \begin{itemize}
% \item greene+2013, Wardle Yusef-zadeh 12, greene+10, schulz henkel 03, tarchi+12
% \item rotation curves implied by the disk (and the small compact scales involved) give direct measurement of the SMBH masses
% \item greene+10 model the maser rotation curves to derive BH masses; use SDSS to derive stellar velocity dispersions -- to very high precision!! (<15\%)
% \item find that maser galaxies fall below the $M_BH-\sigma$ relation defined for ellipticals -- the relation cannot be used to derive the BH mass function for AGN -- NO UNIVERSAL power-law MBH-sigma relation
% \item Find M_BH relates to properties of the bulge, but not strongly with overall galactic luminosity
% \item Combined with full rotation curves of these galaxies (ie from HI), can check for correlation of M_BH w/ mass of dark matter halo (Ferrarese 02, at least for non-spiral galaxies)
% \item greene+13 explore using MM disks to find BH accretion
% \item use HST/WFC3 to observ 9 MM disk galaxies; young stars seen, MM disks do not align w/ circumnuclear disks or bars of stars
% \item disk preferentially aligned w/ radio continuum jets on up to 50 pc scales
% \item ALMA will allow to search the gas component directly at sub-" resolution
% \item favour misalignment scenario due to accretion changing angular momentum as a function of scale
% \item the misalignment between the 2 components may be related to accretion on the BH -- disk warping from radiation pressure (pringle 97),angular momentum changes w/ scale due to accretion events (hopkins \& quataert 10, 11); disk alignments boost accretion rate (nixon+12) - gas angular momentum dissipated where disks meet, could boost accretion rate by 1 dex; gas from external sources giving accretion at random angles from small satellites
% \item Paper III kuo+11 - derive masses for 7 SMBHs (including UGC 3789) -- all consistent w/ edge-on circumnuclear disks w/ inner radii b/w 0.09-0.5 pc -- consistent keplerian motion
% \item mass densities w/in maser disk b/w $0.12-60\times10^{10}$ Msol/pc3; masses b/w $0.76-6.5\times10^7$ Msol
% \item The BH mass from virial arguments agrees w/ 4 galaxies -- have much larger errors though -- method from Greene ho 06, kim+08, vestergaard osmer 09 -- "f"-value of $5.2\pm1.3$  woo+10 where f is an empirically determined factor depending on structure, kinematics, orientation of broad-line region
% \item Large improvements to constraining the $M_BH-\sigma$ relation at the low end -- evidence for not a single power law as was originally proposed (see also the greene+11,13 results)
% \item Using constraints w/ Plummer model for dense clusters of compact objects -- rule out clusters as the dominant mass source in the nuclei, best explanation is SMBHs
% \end{itemize}

% \subsection{Jet masers}
% \label{sub:h2o_jets}

% \begin{itemize}
% \item tarchi+12
% \item interaction b/w radio jet and MCs along path; or amplification of bkg radio continuum from jet through foreground cloud
% \item display single broad emission feature -- two examples are Mrk328 NGC1052 (Falcke+2000, braatz+94)
% \item maser located along radio continuum of jet, separate from nucleus position in both cases (pek+03 1052, claussen+98 328)
% \item peck+03 used "reverberation mapping techniques" to constrain jet properties (density and velocity) -- maser from shock region b/w jet and MC
% \item in 1052, sawada-satoh+08 find maser occurring in foreground of jet -- cloud amplifying radio jet emission in circumnuclear torus
% \item maser where free-free opacity of thermal plasma absorbing synch rad is large
% \item contraction of cloud towards nucleus gives redshifted features
% \item 
% \end{itemize}

% \subsection{Outflow Masers}
% \label{sub:h2o_outflows}

% \begin{itemize}
% \item tarchi+12, greeenhill+03, mccallum+05
% \item Only detected in Circinus -- seyfert 2 
% \item 0.2 pc radius disk where masers located (tristram+07), also a thick torus out to pc scales, slightly cooler than disk, show a clumpy distribution in torus
% \item evidence for collimated AGN outflow w/ maser position
% \item mccallum+03 confirm scintillation of H2O mm, rapid variability linked to interstellar scintillation seen in many flat spectrum AGN
% \item orginal variation obersved by greenhill+97
% \item can test for annual cycles in scintillation, constraining the peculiar velocity of the scattering materialm estimate anisotropies in the medium -- create model of the source
% \item can also lead to direct measurement of ISM velocity -- important for constraining any model of anisotropic structure
% \item weak scintillation model explains observed variability assuming a local screen; distant, strongly scattering screen model fits observations, shows unseen anti-correlation b/w modulation index and timescale
% \end{itemize}

\subsection{Measuring Hubble's Constant}
\label{sub:h2o_cosmo}

The Megamaser Cosmology Project (MCP) is an on-going ambituous project to provide a high-precision measurement of Hubble's Constant ($H_0$), independent of previous estimates \citep{reid2009_mmproject_I}. The project relies on the ability to measure the distances to known H$_2$O mega-maser hosts with Keplerian circumnuclear disks to high precision (\S\ref{sub:h20_agn}). To reach uncertainties in $H_0$ less than 10\% for each maser hosting galaxy, a significant observational time investment is require from multiple facilities. Multy-epoch observations are conducted with single dish facilities (GBT \& Arecibo) on a near monthly basis to constrain the centripetal acceleration, $dV/dt$. These are then combined with a VLBI observation (sometimes multiple), which constrains the angular acceleration of the maser spots, $dV/d\theta$.

The MCP has now applied their technique for measuring the distance and $H_0$ for four galaxies: UGC 3789 \citep{Braatz_2010,reid2013_mmproject_IV}, NGC 6264 \citep{Kuo_2013}, NGC 6323 \citep{kuo2015_mmproject_VI}, and NGC 5765b \citep{gao2015}. All of these measurements required multi-epoch observations, along with modeling several aspects of the disks themselves in order to measure the desired parameter $H_0$. In total, 10 parameters need to be constrained: $H_0$, black hole mass, black hole line-of-sight velocity, the peculiar velocity and 6 parameters constraining the disk. \citet{reid2013_mmproject_IV} used Metropolis-Hastings sampling (MCMC) to simultaneously fit the entire model with all of the available observational data. For all four galaxies, this technique provided well-constrained posterior distributions and excellent Markov Chain diagnostics (low autocorrelation, $\sim 30\%$ acceptance rate of samples, etc...). They find values of $H_0=68.9\pm7.1$ km/s/Mpc for UGC 3789, $H_0=68\pm9$ km/s/Mpc for NGC 6264, $H_0=73^{+26}_{-22}$ km/s/Mpc for NGC 6323, and $H_0=66.0\pm6.0$ km/s/Mpc for NGC 5765b, which all agree within the quoted 1$\sigma$ errors from the posterior distributions. The expected uncertainties are near the target 10\% expected for the survey \citep{reid2013_mmproject_IV}, except for NGC 6323 which has a significantly lower surface brightness and is near the sensitivity limits of current instrumentation \citep{kuo2015_mmproject_VI}. To reach the 3\% uncertainty limit of other experimentally derived values of $H_0$, the MCP project estimated they would need to perform this analysis for 10 galaxies, each with an individual uncertainty limit of $\sim 10\%$ \citep{reid2013_mmproject_IV}.  

Constraining $H_0$ is important for constraining the $w$ parameter for the equation of state for dark energy \citep{reid2013_mmproject_IV}. The constraints derived by \citet{reid2013_mmproject_IV} are combined with the WMAP7 results in Figure \ref{fig:h20_H0_w} (red contours). Shaded regions in the figure are from the WMAP results alone, and the blue contours show the expected constraints available upon completion of the MCP. The combined WMAP7 and \citet{reid2013_mmproject_IV} results give $w=-0.98\pm0.20$. The constraints from the completed MCP are expected to reduce the uncertainty to $\pm0.10$. By utilizing the newest Planck results, it may possible to further constrain $w$ beyond these expectations.

% \begin{itemize}
% \item Paper VII pesce+15 -- testing disk physics; test maoz & mckee 98 model for inversion in gas behind spiral shocks in a disk (results from Humphreys+08)
% \item use 16 galaxies that have VLBI mapping at 22 GHz (9 show keplerian motion) 
% \item find evidence of scintillation, likely due to ISM (just like that seen in the OH in Circinus); favour a local screen at $~70$ pc
% \item constrain B-field based on non-detections of Zeeman splitting during flaring events (down to $~73$ mG, mostly around 200-300 mG)
% \item maoz & mckee 98 seeked model to explain NGC 4258 where redshifted components are much brighter than the blue. Model says that inversion only occurs due to the post-shock gas behind a spiral shock in the accretion disk (viewed edge-on)
% \item trailing shocks (ie. redshifted) are then preferential for masers
% \item show, via likelihood analysis, that NGC 4258 is the only significant outlier in the sample (assuming many gaussians)
% \item find a slight favouring for the MM08 model (not very convincing, but...); tested by comparing the drifts of the blue and red shifts, which should be different according to the model
% \item Time variability -- Long term (hundreds of days) is `bulk variability' seen in all sufficiently monitored galaxies (dynamical timescale for average maser disk is $~10^4$ years so this must arise from much closer to the AGN than the observed masers; `reverberation mechanism' of gallimore+01); short-term ($~$monthly) showing significant flaring of individual maser lines (may be due o chance alignments of masing clumps in the disk (kartje+99) assuming masing comes from individual clumps in the disk)
% \item extremely short variations ($~$day, in NGC 3079 (vlemmings+07) amd circinus (mccallum+05)) likely due to ISM scintillation. Show that ESO 558-G009 also shows this in this paper
% \item Claussen + Lo 86 note systematic variations from nucleart maser s in NGC 1068, suggesting all are powered by a common source (nucleus)
% \item Then change would reverberate radially throughout the disk
% \item Gallimore+01 argued that there is a correlation b/w the red and blue emission in the disk for 1068
% \item Create a radial spectrum to test this for other targets, if BH mass constrained to better than 10\% (6 of the samples)
% \item Find no correlation for the samples down to their sensitivity limit of $~$10s of mJy
% \item Zeeman effect for H2O - B-fields in AGN thought to be important; MRI likely way angular momentum is transported in accretion disks (balbus hawley 91); launching outflows (blandford payne 82, keating+12)
% \item 22 GHz transition made up of up to 6 fine structure transitions (fiebig guesten 89) -- is non-paramagnetic but Zeeman splitting arises from coupling b/w nucleus magnetic moment and external field; nearly $10^3$ time weaker than in OH (don't have unpaired electron to couple w/ field)
% \item derive upper limits, based primarily on modelling of modjaz+05
% \item limits typically $~3-50$ times greater than values detected in OH (robishaw+08, mcbride+13), but is regime where magnetic pressure is comparable to the gas pressure in the disk
% \item focussed testing during flaring (as that gives the best S/N)
% \end{itemize}

% \subsection{Connection between OH and H$_2$O?}
% \label{sec:oh_and_h2o}

% This section and the previous (\S\ref{sec:OH}) have argued that OH mega-masers arise from starburst activity and H$_20$O mega-masers are inherently linked to AGN activity. Are these two groups mutually exclusive or not? Since H$_2$O mega-masers are found on parsec-scales from the nucleus and OH mega-masers are distributed over 100-pc scales, it is certainly plausible that both types could be produced in a single galaxy. Until recently, the connection was hindered due to a near mutual exclusion of survey targets for the respective species. \citet{darling2006} studied the optical spectra of the confirmed OH mega-maser hosts from their earlier study \citep{darling2002_paperIII} and found that 33\% are starbursts, 42\% are LINER galaxies, and the remaining 25\% are Sy2 galaxies. Thus, there appears to be a significant overlap in potential mega-masing targets for both OH and H$_2$O.



% \begin{itemize}
% \item wiggins+15, tarchi+12, wagner+13
% \item Two cases connecting OH MM w/ AGN Mrk321 \& IIIZw 35
% \item Mrk321 -- from VLBI, OH maps rotating molecular torus (klockner+03) 30-100 pc from nucleus
% \item IIIZw 35 -- pc-size OH clouds of edge-on torus (pihlstrom+01) - has diffuse and compact components in ring structure amplifying bkg continuum (parra+05)
% \item Arp 299 appears to show OH emission from the nucleus of IC 694 (baan 85); H2O MM emission as well (tarchi+07a) slight offset from OH in a nuclear outflow (Tarchi+11a)
% \item 5 other galaxies show masers in both, but are not MMs (includes NGC 253 and M82, nearby starbursts, Tarchi+11a)
% \item wiggins+15 searches for 22 Ghz emission in OH MM galaxies w/ GBT -- second detection in IIZw96, also a merger
% \item sample from tarchi+10 of OH emitters visible at GBT that had never been searched for 22 GHz emission -- 47 objects observed
% \item IIZw 96 shows high luminosity, narrow features in water; no high velocity shifts so not associated w/ a disk -- close resemblance to Arp 299 water line; features consistent with pumping from the AGN
% \item suggestive of a brief phase in galaxy mergers where maser coexistance is possible
% \item evidence also possibly shows exclusion for kilo-masers in OH MM hosts
% \item tentative detection of IRAS 15179+3956, another merging system
% \end{itemize}