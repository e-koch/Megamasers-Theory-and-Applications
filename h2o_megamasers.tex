\section{H_2O Megamasers}
\label{sec:h2o_mm}

Multiple H$_2$O transitions have been observed producing maser emission in galactic source \citep{Elitzur_1992}, however the dominant transition detected as stimulated emission is the 22 GHz transition (see Table \ref{tab:maser_props}). This transition is relatively easy to cause stimulated emission in since it corresponds at a `backbone' transition (lowest levels at each rotational $J$-value). These transitions have large Einstein $A$-values, leading to trapping of the emitted photons \citep{stahler_palla_2004}. This makes collisional excitement and de-excitement more effective than radiation in producing these transitions. The environment needed to produce the 22 GHz H$_2$O emission must pump molecules to fairly high energy levels (640 K, Table \ref{tab:maser_props}). As shown by \citet{stahler_palla_2004}, shocked regions are capable of providing collisional excitement to these levels, and indeed this is where H$_2$O maser emission is observed in both galactic and extragalactic source \citep{Elitzur_1992, lo2005}. 

The 22 GHz H$_2$O maser emission is associated with circumstellar material around late-type stars, and with molecular outflows due to young, embedded stars. These environments suggest a collisional pumping mechanism. H$_2$O mega-maser emitting regions are typically found within a few parsecs of an AGN, coinciding with regions of dense molecular gas heated to suitable temperatures for masers to occur. 

\begin{itemize}
\item bennert+09, tarchi+12, tarchi+07, cernicharo+06, baan+06, baan klockner 05, baan+82, dos santos lepine 79
\item 
\end{itemize}

\subsection{Extragalactic Mega-maser Properties}
\label{sub:h2o_props}

Extragalactic

\begin{itemize}
\item sales+15, wagner+13, raluy+1998
\item circumnuclear and nuclear jet driven masers and outflow
\item Detection rates: 150/3000 searched galaxies show H2O MM (tarchi+2012)
\item majority are Seyfert 2 or LINERs (z<0.05)
\item much higher detection rates simply among these 2 classes (~25\%, tarchi+12)
\item AGN w/ H2O MM show high $N_H > 10^23 cm^-2$, some are compton-thick ($>10^24$)
\item rough correlation w/ unabsorbed X-ray luminosities (kondratko+06) -- uncertain due to poor X-ray data for 1/2 of MM detections
\item Properties of AGN w/ MM: primarily in X-ray absorbed source (zhang+10, ramolla+11, zhu+11); have larger radio luminosities (zhang+12), sample based on extinction-corrected [OIII]5007 flux (zhu+11)
\item highest occurence in Seyfert 2 galaxies -- not suprising since Unified model as it predicts an edge-on disk or torus (ie giving longest path length), tarchi+12
\item Some Seyfert 1 detections have occurred -- more unclear since Unified model predicts these should be face-on
\item Only 1/150 searches of pure Sy1 have given detection, but seem to occure more often with narrow line Sy1's (narrowst balmer lines, strongest FeII emission, exteme X-ray properties (komossa 08))
\item 5/71 detections in this subclass (tarchi+11b) (~7\% rate), sample is volume-limited (those within 1e4 km/s boost rate to ~24\%, highest rate of any AGN class)
\item many proposed reasons: intermediate viewing angles, accretion rates near eddington, strong outflows (tarchi+12)
\item No detections in elliptical or radio-loud galaxies -- several surveys conducted (see refs in Tarchi+12); seems regardless of the AGN type
\item however, there are 4 unclassified H2O MM galaxies which show unique properties (5 including NGC 1052)
\item Could be due to: lack of molecular gas (henkel+98); instability from tidal disruption in clouds orbiting particularly massive SMBHs (tarchi+07b); insufficient sensitivity of completed surveys
\end{itemize}

\subsection{AGN: Disks \& SMBH masses}
\label{sub:h20_smbh_mass}

\begin{itemize}
\item greene+2013, Wardle Yusef-zadeh 12, greene+10, schulz henkel 03, tarchi+12
\item these are circumnuclear masers
\item Unified model of AGN (antonucci 93, urry+padovani 95) predicts an accretion disk about AGN surrounded by torus or thick disk
\item Type 1 AGN - view accretion disk and BH through hole in torus; type 2 - direct view of nucleus obscured by torus
\item masers are unique tool to probe ~pc regions around AGN -- other bands are highly obscured
\item characteristically have triple peak in velocity -- center at systematic vel of galaxy, w shifted by hundreds of km/s -- rotation of disk
\item NGC 4258 best studied for this  (Miyoshi+95, Herrnstein+99) -- MCP has greatly extended this
\item Disks appear to be thin and keplerian, extending to sub-pc distances from the AGN
\item rotation curves implied by the disk (and the small compact scales involved) give direct measurement of the SMBH masses
\item greene+10 model the maser rotation curves to derive BH masses; use SDSS to derive stellar velocity dispersions -- to very high precision!! (<15\%)
\item find that maser galaxies fall below the $M_BH-\sigma$ relation defined for ellipticals -- the relation cannot be used to derive the BH mass function for AGN -- NO UNIVERSAL power-law MBH-sigma relation
\item Find M_BH relates to properties of the bulge, but not strongly with overall galactic luminosity
\item Combined with full rotation curves of these galaxies (ie from HI), can check for correlation of M_BH w/ mass of dark matter halo (Ferrarese 02, at least for non-spiral galaxies)
\item wardle yusef-zadeh 21 derive a relation b/w MM disk radius and BH mass
\item argue that disks are formed through capture of MCs passing through galactic nucleus (w/in some impact parameter)
\item Find $R=4G\lambda^2M/v^2$, where $\lambda$ is the fraction of angular momentum reduced when cloud enters the disc, v is cloud velocity, M is BH mass
\item matches observational values
\item greene+13 explore using MM disks to find BH accretion
\item use HST/WFC3 to observ 9 MM disk galaxies; young stars seen, MM disks do not align w/ circumnuclear disks or bars of stars
\item disk preferentially aligned w/ radio continuum jets on up to 50 pc scales
\item ALMA will allow to search the gas component directly at sub-" resolution
\item favour misalignment scenario due to accretion changing angular momentum as a function of scale
\item the misalignment between the 2 components may be related to accretion on the BH -- disk warping from radiation pressure (pringle 97),angular momentum changes w/ scale due to accretion events (hopkins \& quataert 10, 11); disk alignments boost accretion rate (nixon+12) - gas angular momentum dissipated where disks meet, could boost accretion rate by 1 dex; gas from external sources giving accretion at random angles from small satellites
\end{itemize}

\subsection{Jet masers}
\label{sub:h2o_jets}

\begin{itemize}
\item tarchi+12
\item interaction b/w radio jet and MCs along path; or amplification of bkg radio continuum from jet through foreground cloud
\item display single broad emission feature -- two examples are Mrk328 NGC1052 (Falcke+2000, braatz+94)
\item maser located along radio continuum of jet, separate from nucleus position in both cases (pek+03 1052, claussen+98 328)
\item peck+03 used "reverberation mapping techniques" to constrain jet properties (density and velocity) -- maser from shock region b/w jet and MC
\item in 1052, sawada-satoh+08 find maser occurring in foreground of jet -- cloud amplifying radio jet emission in circumnuclear torus
\item maser where free-free opacity of thermal plasma absorbing synch rad is large
\item contraction of cloud towards nucleus gives redshifted features
\item 
\end{itemize}

\subsection{Outflow Masers}
\label{sub:h2o_outflows}

\begin{itemize}
\item tarchi+12, greeenhill+03, mccallum+05
\item Only detected in Circinus -- seyfert 2 
\item 0.2 pc radius disk where masers located (tristram+07), also a thick torus out to pc scales, slightly cooler than disk, show a clumpy distribution in torus
\item evidence for collimated AGN outflow w/ maser position
\item mccallum+03 confirm scintillation of H2O mm, rapid variability linked to interstellar scintillation seen in many flat spectrum AGN
\item orginal variation obersved by greenhill+97
\item can test for annual cycles in scintillation, constraining the peculiar velocity of the scattering materialm estimate anisotropies in the medium -- create model of the source
\item can also lead to direct measurement of ISM velocity -- important for constraining any model of anisotropic structure
\item weak scintillation model explains observed variability assuming a local screen; distant, strongly scattering screen model fits observations, shows unseen anti-correlation b/w modulation index and timescale
\end{itemize}

\subsection{Cosmology: Distance Determination}
\label{sub:h2o_cosmo}

\begin{itemize}
\item Megamaser comology project papers
\item goal to measure hubble constant, improve extragalactic scale and constrain nature of dark energy
\item search for H2O from AGN w/ sub-pc accretion disks (w/ GBT/Arecibo), then follow up w/ VLBI
\item Paper I  reid+09 - VLBI map in UGC 3789 - rotational velocities of ~800 km/s to radii of 0.08 pc
\item MM discovered by Braatz gugliucci 08
\item edge-on Keplerian rotation about $10^7 M_sol$ SMBH
\item well into Hubble flow -- $v_rec\approx3325$ km/s
\item Paper II braatz+10 - derive distance to UGC 3789, determine value for Hubble constant
\item measure angular-diameter distances using maser emission in the circumnuclear flow
\item use 3.2 years of monthly GBT observations to measure the centripetal accelerations of the maser components
\item find evidence for maser components confined into 2 rings -- determine distance of $49.0\pm7.0$ Mpc -- most accurate geometric measurement of a galaxy in the Hubble flow -- using 2 thin ring model
\item Gives $H_0=69\pm11$ km/s/Mpc; $M_{BH}=1.09\times10^7 \pm 14% $ Msol -- use standard $\Lambda$CDM model with $\Sigma_m=0.26$
\item Main uncertainty comes from orbital curvature parameter $\Omega$ and uncertainties from deriving the centripetal accelerations (lowered w/ more observations), will decrease to sub-10\% errors on distance.
\item Paper III kuo+11 - derive masses for 7 SMBHs (including UGC 3789) -- all consistent w/ edge-on circumnuclear fisks w/ inner radii b/w 0.09-0.5 pc -- consistent keplerian motion
\item mass densities w/in maser disk b/w $0.12-60\times10^{10}$ Msol/pc3; masses b/w $0.76-6.5\times10^7$ Msol
\item BH mass errors at ~11\%, dominated by the errors in the Hubble constant -- used $H_0=73\pm8$ km/s/Mpc from freedman+01
\item The BH mass from virial arguments agrees w/ 4 galaxies -- have much larger errors though -- method from Greene ho 06, kim+08, vestergaard osmer 09 -- "f"-value of $5.2\pm1.3$  woo+10 where f is an empirically determined factor depending on structure, kinematics, orientation of broad-line region
\item Large improvements to constraining the $M_BH-\sigma$ relation at the low end -- evidence for not a single power law as was originally proposed (see also the greene+11,13 results)
\item Using constraints w/ Plummer model for dense clusters of compact objects -- rule out clusters as the dominant mass source in the nuclei, best explanation is SMBHs
\item Paper IV reid+13 -- Directly measure $H_0=89.9\pm7.1$ km/s/Mpc (10\% uncertainty)
\item based on continued observations of UGC 3789 in papers I-II, with modelling improvements
\item completely independent of other methods for determining $H_0$
\item Then is at a distance of $49.6\pm5.1$ Mpc, $M_BH=1.16\pm0.12\times10^7$ Msol
\item fitted centripetal accelerations in 2 ways: (1) by-eye identification of peaks that were followed through the time-series; (2) random choice of initial values, which are lsq fit. Repeated 100 times and best 4 (based on reduced $\chi^2$) were chosen -- GOOD agreement b/w the 2
\item Globally fit 10 parameters to using the data using MCMC -- $H_0$ is one parameter, fit a gaussian to the posterior and report error bars based on that
\item Fit includes peculiar velocity, los velocity of BH, $H_0$, BH mass, BH position wrt maser positions, inclination angle, inclination warping, gas orbital eccentricity, angle of periastron wrt to LOS, deriv of ang of peri wrt radii in disk -- flat priors except for peculiar velocity -- excellent convergence
\item Constraining EoS for dark energy -- the "w" term, combine their estimate for $H_0$ w/ WMAP7 results to independently constrain EoS -- gives $w=-0.98\pm0.20$, improving constraints by factor of 2. Best result likely until SKA -- \textbf{COULD SHOW FIGURE 7 OF THERE PAPERS}
\item Paper V kuo+13 - Measure distance to NGC 6264 to be $144\pm19$ Mpc, $H_0=68\pm9$ km/s/Mpc, and $M_BH = 3.09\pm0.42\times 10^7$ Msol
\item same techniques as was used in first 4 papers -- also used Effelsburg
\item measure centripetal accelerations of the masers as they rotate about in the disk
\item repeat bayesian model from paper IV -- required multiple rings to constrain the disk, unlike the 2 ring model used before
\item first direct distance measurement of a galaxy beyond 100 Mpc
\item this represents the limit that current radio telescopes can achieve -- propose extensions to this method to simultaneously fit with other maser emission (if detected...)
\item Paper VI kuo+15 -- perform analysis on NGC 6323, find $H_0=73^{+26}_{-22}$ km/s/Mpc, limited by low SB of detections ($\lt 15$ mJy)
\item specifically develop techniques to help in the low SB regime (until new telescopes, NG-VLA, etc, etc), but then state they don't help much
\item Paper VII pesce+15 -- testing disk physics; test maoz & mckee 98 model for inversion in gas behind spiral shocks in a disk (results from Humphreys+08)
\item use 16 galaxies that have VLBI mapping at 22 GHz (9 show keplerian motion) 
\item find evidence of scintillation, likely due to ISM (just like that seen in the OH in Circinus); favour a local screen at ~70 pc
\item constrain B-field based on non-detections of Zeeman splitting during flaring events (down to ~73 mG, mostly around 200-300 mG)
\item maoz & mckee 98 seeked model to explain NGC 4258 where redshifted components are much brighter than the blue. Model says that inversion only occurs due to the post-shock gas behind a spiral shock in the accretion disk (viewed edge-on)
\item trailing shocks (ie. redshifted) are then preferential for masers
\item show, via likelihood analysis, that NGC 4258 is the only significant outlier in the sample (assuming many gaussians)
\item find a slight favouring for the MM08 model (not very convincing, but...); tested by comparing the drifts of the blue and red shifts, which should be different according to the model
\item Time variability -- Long term (~hundreds of days) is `bulk variability' seen in all sufficiently monitored galaxies (dynamical timescale for average maser disk is $~10^4$ years so this must arise from much closer to the AGN than the observed masers; `reverberation mechanism' of gallimore+01); short-term (~monthly) showing significant flaring of individual maser lines (may be due o chance alignments of masing clumps in the disk (kartje+99) assuming masing comes from individual clumps in the disk)
\item extremely short variations (~day, in NGC 3079 (vlemmings+07) amd circinus (mccallum+05)) likely due to ISM scintillation. Show that ESO 558-G009 also shows this in this paper
\item Claussen + Lo 86 note systematic variations from nucleart maser s in NGC 1068, suggesting all are powered by a common source (nucleus)
\item Then change would reverberate radially throughout the disk
\item Gallimore+01 argued that there is a correlation b/w the red and blue emission in the disk for 1068
\item Create a radial spectrum to test this for other targets, if BH mass constrained to better than 10\% (6 of the samples)
\item Find no correlation for the samples down to their sensitivity limit of ~10s of mJy
\item Zeeman effect for H2O - B-fields in AGN thought to be important; MRI likely way angular momentum is transported in accretion disks (balbus hawley 91); launching outflows (blandford payne 82, keating+12)
\item 22 GHz transition made up of up to 6 fine structure transitions (fiebig guesten 89) -- is non-paramagnetic but Zeeman splitting arises from coupling b/w nucleus magnetic moment and external field; nearly $10^3$ time weaker than in OH (don't have unpaired electron to couple w/ field)
\item derive upper limits, based primarily on modelling of modjaz+05
\item limits typically ~3-50 times greater than values detected in OH (robishaw+08, mcbride+13), but is regime where magnetic pressure is comparable to the gas pressure in the disk
\item focussed testing during flaring (as that gives the best S/N)
\end{itemize}

\subsection{Connection between OH and H$_2$O?}
\label{sec:oh_and_h2o}

\begin{itemize}
\item wiggins+15, tarchi+12
\item OH megamaser seem to be uniquely associated with LIRGs and ULIRGs ($L_IR>10^11 L_sol$) -- ideal for IR radiative pumping 
\item Darling Paper III IR OH correlation
\item Two cases connecting OH MM w/ AGN Mrk321 \& IIIZw 35
\item Mrk321 -- from VLBI, OH maps rotating molecular torus (klockner+03) 30-100 pc from nucleus
\item IIIZw 35 -- pc-size OH clouds of edge-on torus (pihlstrom+01) - has diffuse and compact components in ring structure amplifying bkg continuum (parra+05)
\item Arp 299 appears to show OH emission from the nucleus of IC 694 (baan 85); H2O MM emission as well (tarchi+07a) slight offset from OH in a nuclear outflow (Tarchi+11a)
\item 5 other galaxies show masers in both, but are not MMs (includes NGC 253 and M82, nearby starbursts, Tarchi+11a)
\item wiggins+15 searches for 22 Ghz emission in OH MM galaxies w/ GBT -- second detection in IIZw96, also a merger
\item sample from tarchi+10 of OH emitters visible at GBT that had never been searched for 22 GHz emission -- 47 objects observed
\item IIZw 96 shows high luminosity, narrow features in water; no high velocity shifts so not associated w/ a disk -- close resemblance to Arp 299 water line; features consistent with pumping from the AGN
\item suggestive of a brief phase in galaxy mergers where maser coexistance is possible
\item evidence also possibly shows exclusion for kilo-masers in OH MM hosts
\item tentative detection of IRAS 15179+3956, another merging system
\end{itemize}