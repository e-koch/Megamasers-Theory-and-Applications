\section{H_2O Megamasers}
\label{sec:h2o_mm}

Multiple H$_2$O transitions have been observed producing maser emission in galactic source \citep{Elitzur_1992}, however the dominant transition detected as stimulated emission is the 22 GHz transition (see Table \ref{tab:maser_props}). This transition is relatively easy to cause stimulated emission in since it corresponds at a `backbone' transition (lowest levels at each rotational $J$-value). These transitions have large Einstein $A$-values, leading to trapping of the emitted photons \citep{stahler_palla_2004}. This makes collisional excitement and de-excitement more effective than radiation in producing these transitions. The environment needed to produce the 22 GHz H$_2$O emission must pump molecules to fairly high energy levels (640 K, Table \ref{tab:maser_props}). As shown by \citet{stahler_palla_2004}, shocked regions are capable of providing collisional excitement to these levels, and indeed this is where H$_2$O maser emission is observed in both galactic and extragalactic source \citep{Elitzur_1992, lo2005}. 

The 22 GHz H$_2$O maser emission is associated with circumstellar material around late-type stars, and with molecular outflows due to young, embedded stars. These environments suggest a collisional pumping mechanism. H$_2$O mega-maser emitting regions are typically found within a few parsecs of an AGN, coinciding with regions of dense molecular gas heated to suitable temperatures for masers to occur. 

\begin{itemize}
\item bennert+09, tarchi+12, tarchi+07, cernicharo+06, baan+06, baan klockner 05, baan+82, dos santos lepine 79
\item 
\end{itemize}

\subsection{Extragalactic versus Galactic Properties}
\label{sub:h2o_props}

\begin{itemize}
\item sales+15, wagner+13, mccallum+05, raluy+1998
\item circumnuclear and nuclear jet driven masers
\end{itemize}

\subsection{AGN: Disks \& SMBH masses}
\label{sub:h20_smbh_mass}

\begin{itemize}
\item greene+2013, Wardle Yusef-zadeh 12, greene+10, schulz henkel 03, tarchi+12
\item these are circumnuclear masers
\item Unified model of AGN (antonucci 93, urry+padovani 95) predicts an accretion disk about AGN surrounded by torus or thick disk
\item Type 1 AGN - view accretion disk and BH through hole in torus; type 2 - direct view of nucleus obscured by torus
\item masers are unique tool to probe ~pc regions around AGN -- other bands are highly obscured
\item characteristically have triple peak in velocity -- center at systematic vel of galaxy, w shifted by hundreds of km/s -- rotation of disk
\item NGC 4258 best studied for this  (Miyoshi+95, Herrnstein+99) -- MCP has greatly extended this
\item 
\end{itemize}

\subsection{Jet masers}
\label{sub:h2o_jets}

\begin{itemize}
\item tarchi+12
\item interaction b/w radio jet and MCs along path; or amplification of bkg radio continuum from jet through foreground cloud
\item display single broad emission feature -- two examples are Mrk328 NGC1052 (Falcke+2000, braatz+94)
\item maser located along radio continuum of jet, separate from nucleus pos'n in both cases (pek+03 1052, claussen+98 328)
\item peck+03 used "reverberation mapping techniques" to constrain jet properties (density and velocity) -- maser from shock region b/w jet and MC
\item in 1052, sawada-satoh+08 find maser occurring in foreground of jet -- cloud amplifying radio jet emission in circumnuclear torus
\item maser where free-free opacity of thermal plasma absorbing synch rad is large
\item contraction of cloud towards nucleus gives redshifted features
\item 
\end{itemize}

\subsection{Cosmology: Distance Determination}
\label{sub:h2o_cosmo}

\begin{itemize}
\item Megamaser comology project papers
\item 
\end{itemize}

\subsection{Connection between OH and H$_2$O?}
\label{sec:oh_and_h2o}

\begin{itemize}
\item wiggins+15, tarchi+12
\end{itemize}