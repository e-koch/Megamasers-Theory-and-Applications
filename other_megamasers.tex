\section{Other mega-maser detections}
\label{sec:other}

While the vast majority of detections and observational effort has focussed on detecting OH and H$_2$O mega-masers, upgrades to some radio telescopes have lead to the first detections of CH$_3$OH and SiO mega-masers \citep{wang2014_SiO_CH3OH, chen_methanol_2015}. Both of these species have widespread galactic emission: SiO masers are associated with circumstellar material around late-type stars, and CH$_3$OH masers are commonly observed in star-forming regions of dense molecular gas \citep{Elitzur_1992}. Properties of the SiO line are shown in Table \ref{tab:???}

The first, and only, H$_2$CO mega-maser was reported by \citet{baan1986} towards IC 4553 (in the well-studied Arp 220 system). IC 4553 was the first detection of on OH mega-maser. \citet{baan1986} find similar velocity components in between the two mega-maser detections. It's possible that H$_2$CO mega-masers coincide with other OH mega-masers, however they may be too faint to observe with current instrumentation. The flux density of the H$_2$CO maser is nearly 100 times fainter than the OH detection. Too little is known about this class of mega-masers to infer an accurate relation to OH mega-masers.

The first detections of SiO and CH$_{3}$OH mega-masers are reported by \citet{wang2014_SiO_CH3OH} in observations towards NGC 1068 using the IRAM 30m telescope. This source has a previously discovered H$_2$O mega-maser \citep{Gallimore_2001}. The velocities components of the SiO nearly match those of the H$_2$O emission, indicating that the SiO likely also arises due to the compact, nuclear disk (see \S\ref{sub:h2o_props}). The CH$_3$OH exhibit different velocities and linewidths than the SiO emission. The author's propose that this emission arises from a shock front associated with a nuclear jet or a molecular outflow. Future high-resolution observations with ALMA may allow for AGN feedback to be studied from the nuclear disk out to $\sim 100$ pc using maser emission of these three species. If SiO mega-masers do, in fact, trace the nuclear disk as the H$_2$O ones do, they may be used for the same applications discussed in \S\ref{sec:h2o_mm}. 

A second detection of CH$_3$OH mega-masers was recently reported by \citet{chen_methanol_2015} toward the well-studied Arp 220, which is also known to harbour OH and H$_2$CO mega-maser emission. They report a detection of CH$_3$OH in the same 36.2 GHz line as \citet{wang2014_SiO_CH3OH} and at 37.7 GHz using the Australia Telescope Compact Array. The observed spectral features are narrow and their inferred luminosities are about 8 orders of magnitude greater than typical galactic CH$_3$OH masers, putting them nearly in the {\it giga}-maser regime. They also conclude that this emission arises due to shock fronts associated with outflows from the nuclear region. They find the position of the mega-masers correlate well with H$\alpha$ emission. These results suggest that CH$_3$OH mega-masers may trace larger scales than OH and H$_2$O mega-masers. Through combined detections of these three species in a system, the influence of the circumnuclear starburst on the surrounding regions could be tested.

Both detections of CH$_3$OH mega-masers come after a survey completed by \citet{darling2003_CH3OH}. The author's observed 28 known OH mega-maser hosts with Arecibo with the C-band receiver, which unfortunately limited the survey targets to redshifts greater than 0.11 due to the frequency cut-off of the receiver. The recent detections explained above are at much smaller redshifts (0.018 for Arp 220, and 0.004 for NGC 1068). It seems likely that the observations of \citet{darling2003_CH3OH} did have adequate sensitivities to find a detection.

These detections will likely spur further surveys for further detections of these masing species. As \citet{darling2012} points out, the first mega-maser detections were serendipitous, and given the widespread nature of galactic strength masers, exploratory searches will likely result in new discoveries of extragalactic masing species. If the nature of these galaxies is in any way similar to the Milky Way, star-forming galaxies should be teeming with masers, kilo-masers, and other mega-masers. 