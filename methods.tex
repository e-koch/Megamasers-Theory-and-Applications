\section{Maser Theory}
\label{sec:maser_theory}

All maser emission arises when photons of specific frequencies incite an excited molecule to emit a photon of the same frequency. This causes a cascade of photons to be emitted within a region of excited molecules, resulting in exponential growth of the intensity of a ray of light. XXX A maser requires XXX A significant portion of maser theory is presented and derived in a series of papers by Moshe Elitzur \citep{Elitzur_1990, Elitzur_1990_paperI, Elitzur_1990_paperII, Elitzur_1991}.

In this section, I consider a two-level system of molecular energy levels and derive this exponential growth in the ray intensity in terms of the emitting region's physical properties, and the properties of the emitting molecule. It should be noted that a proper treatment of any stimulated emission process requires a minimum of three energy levels, as is shown in the thorough derivations by \cite{Elitzur_1992} and \cite{Gray_2009}. Here, I largely follow the simplified derivations presented in Chapter 4 of \citet{Elitzur_1992} and Chapter 14 of \citet{stahler_palla_2004}, which implicitly treat levels unassociated with the maser transition as generic gain and loss rates.

Let there be an upper and lower states, such that the maser results from the transition from the upper to the lower. From the Boltzmann distribution, the excitation temperature is defined by
\begin{equation}
\label{eq:level_pops}
\frac{n_u}{n_{l}} = \frac{g_{u}}{g_{l}} \mathrm{exp}\left( \frac{\Delta E}{kT_{\mathrm{ex}}} \right)
\end{equation}
$n_u$ and $n_l$ are the number densities in the respective upper and lower states, $g_u$ and $g_l$ are the degeneracies of those states, and $\Delta E$ is the transition energy between them. The population levels become inverted when $n_\mathrm{u}/g_\mathrm{u} \gt n_\mathrm{l}/g_\mathrm{l}$, which corresponds to a negative $T_\mathrm{ex}$. Maintaining an inverted population requires an energy source, referred to as the {\it pump} (see XXX ADD SUBSECTION XXX). Transistions in the microwave and radio require the smallest pump action, since $n_\mathrm{u}/g_\mathrm{u} \approx n_\mathrm{l}g_\mathrm{l}$ for kinetic temperatures of a few hundred kelvin, even when in thermal equilibrium. The role of the pump is to excite molecules into the upper and lower states of the maser transition, and is typically achieved through a collisional or radiative process (see XXX add subsection XXX). Without regard for the details of how the pump achieves this, let $P_\mathrm{u}$ and $P_\mathrm{l}$ be the rate per unit volume at which the pump adds molecules to the upper and lower states from other states of the molecule, respectively. Both states may also decay or be excited further into other energy states. Since this must depend on the population in the upper and lower levels, these loss rates per unit volume are $n_u\Gamma_u$ and $n_l\Gamma_l$, where $\Gamma$ is the decay/excitation rate in each level. The level populations of the upper and lower states also depend on the Einstein A and B coefficients, representing spontaneous decay and absorption between the states, in the presence of a radiation field, $\bar{J}$, the mean intensity. Finally, the molecules may collide with other molecules, giving collisional excitation ($\gamma_{lu}$) and deexcitation ($\gamma_{ul}$) rates. A schematic of the processes discussed here are presented in Figure \ref{fig:energy_diagram}.

While observed masers may vary on time-scales ranging from days to months \citep[e.g., ][\S XXX]{Elitzur_1992_review}, I make the assumption that the maser mechanism is generally in a steady-state. The steady state populations of the upper and lower levels are then given by,
\begin{equation}
\label{eq:ss_pops}
0 = P_u - n_u \Gamma_u - \left( n_u B_{ul} - n_l B_{lu} \right)\bar{J} - \left( n_u \gamma_{ul} - n_l \gamma{lu} \right) n_{\mathrm{tot}} - n_u A_{ul} \\
0 = P_l - n_l \Gamma_l + \left( n_u B_{ul} - n_l B_{lu} \right)\bar{J} + \left( n_u \gamma_{ul} - n_l \gamma{lu} \right) n_{\mathrm{tot}} + n_u A_{ul}
\end{equation}
Here, $n_{\mathrm{tot}}$ is the number density of the background gas. The $B$-coefficients are related by $g_lB_{lu}=g_uB_{ul}$, and the collisional coefficients approximately follow $\gamma_{ul}g_u \approx \gamma_{lu}g_l$, since $T_{\mathrm{kin}} \gg \Delta E / k_{\mathrm{B}}$ for the relevant energy transitions for masers \citep{stahler_palla_2004}.

The relevant quantity for understanding the radiative transfer of a maser is the {\it degree of inversion} of the population levels. To derive this, \citet{stahler_palla_2004} make two simplifying assumptions: the loss rates are equal ($\Gamma=\Gamma_u=\Gamma_l$), and, in practice, spontaneous emission is negligible compared to the loss rates for the transitions of interest($\Gamma \gg A_{ul}$) \citep{Elitzur_1992}. Assuming equal loss rates does not change the form of the final result of the derivation. Using these assumptions, and the aforementioned relation, the population inversion can be described in terms of the properties of the two states:
\begin{equation}
\label{eq:pop_inverse_full}
\Delta n \equiv \frac{n_u}{g_u} - \frac{n_l}{g_l} \\
    = \frac{P_u/g_u - P_l/g_l}{\Gamma + \left( 1 + g_u/g_l \right)\gamma_{ul} n_{\mathrm{tot}}}
\end{equation}
This difference, from the Einstein theory, is the criterion for stimulated action \citep{Gray_2009}. As is done in \citet{stahler_palla_2004}, I define the population inversion due to only collisional processes (ie. $\bar{J}=0$), as $\Delta n^{\circ}$. Equation \ref{eq:pop_inverse_full} can then be expressed as:
\begin{equation}
\label{eq:pop_inverse}
\Delta n = \frac{\Delta n^{\circ}}{1 + \bar{J}/\bar{J}_s}
\end{equation}
$\bar{J}_s$ is the {\it saturation intensity}, representing the intensity where the transition will begin to become thermalized due to collisional excitation:
\begin{equation}
\label{eq:sat_intensity}
\bar{J}_s \equiv \frac{\Gamma + \left( 1 + g_u/g_l \right)\gamma_{ul}n_{\mathrm{tot}}}{\left( 1 + g_u/g_l \right)B_{ul}}
\end{equation}
Equation \ref{eq:pop_inverse} shows that the population inversion is not affected greatly by radiation ($\bar{J}$), until it exceeds $\bar{J}_s$ and quickly declines. This makes the maser mechanism self-limiting, since $\bar{J}$ is amplified by the process. The value of $\bar{J}_s$, as will be shown below, sets the maximum intensity that the maser mechanism can produce, and distinguishes between the unsaturated and saturated regimes \citep{Elitzur_1992}.

For spectral line emission, the emission and absorption coefficients are related to the Einstein $A$ and $B$ values of the transition: $j_\nu = h \nu_0 n_u A_{ul}\phi(\nu)$ and $\alpha_\nu = h\nu_0 \left( n_l B_{lu} - n_u B_{ul} \right)\phi(\nu)/4\pi$, respectively. Here, $\phi(\nu)$ is a normalized Doppler profile, corresponding to the ambient temperature, and $\nu_0$ is the central frequency. The radiative transfer equation then has a form of,
\begin{equation}
\label{eq:rad_trans_specific}
\frac{dI_{\nu}}{ds} = -\alpha_{\nu}I_{\nu} + j_{\nu}
    = \frac{h \nu_0}{4\pi} \left[ \Delta n g_u B_{ul} I_{\nu} + n_u A_{ul} \right] \phi(\nu)
\end{equation}
where I have used the relation of the Einstein $B$ coefficients shown before to convert to $\Delta n$ in $\alpha_{\nu}$. Integrating over frequency, and substituting in Equation \ref{eq:pop_inverse}, yields
\begin{equation}
\label{eq:rad_trans}
\frac{dI}{ds} = \frac{h\nu_0}{4\pi \Delta \nu} \left[ \frac{\Delta n^{\circ} g_u B_{ul}}{1 + \bar{J}/\bar{J}_s} I + n_u A_{ul} \right]
\end{equation}
where $I \equiv \int I_{\nu}\phi(\nu)d\nu$, and $\Delta \nu \equiv \int I_{\nu}d\nu/I$ is the effective bandwidth. To solve this equation, the forms of $\bar{J}$ and $n_u$ must be specified. Masers are tightly beamed (see \S XXX REF SUBSECTION XXX), so $\bar{J} \approx I\Delta\Omega/4\pi$, assuming $I$ is constant across $\Delta\Omega$. Then, $\bar{J}$ may be eliminated from Equation \ref{eq:rad_trans}, since $\bar{J}/\bar{J}_s = I/I_s$. The second term in Equation \ref{eq:rad_trans}, containing $n_u$, arises due to spontaneous emission ($A_{ul}$). This contribution will be significantly smaller than the stimulated emission, however it plays the vital role for the maser process in the absence of background radiation. Because of this, I follow \citet{stahler_palla_2004} in approximating $n_u \approx n_u^{\circ}$ to be the population level in the upper state in the absence of radiation. These assumptions simplify Equation \ref{eq:rad_trans} into
\begin{equation}
\label{eq:final_rad_trans}
\frac{dI}{ds} = \frac{I}{L (1+I/I_s)} + \beta \frac{I_s}{L}
\end{equation}
where $L \equiv 4 \pi \Delta \nu / h \nu_0 \Delta n^{\circ} g_u B_{ul}$ is the {\it unsaturated growth length}, and
\begin{equation}
\label{eq:beta}
\beta \equiv \frac{n_u^{\circ}}{g_u\Delta n^{\circ}} \left( \frac{\Delta\Omega}{4\pi} \right) \frac{A_{ul}}{B_{ul}\bar{J}_s}
\end{equation}
is a small dimensionless quantity \citep{stahler_palla_2004}. The second and third terms in Equation \ref{eq:beta} can be shown to be small using previous arguments and Equation \ref{eq:sat_intensity}, while the first term is limited by the population inversion and should always be near unity.

Solving Equation \ref{eq:final_rad_trans} in the limit of small $\beta$ yields,
\begin{equation}
\label{eq:full_solution}
i(s) - i_0 + \mathrm{ln}\left( \frac{i(s) + \beta}{i_0 + \beta} \right) = \frac{s}{L}
\end{equation}
where I have defined $i(s) = I(s)/I_s$ and $i_0 = I(0)/I_s$; the background radiation is represented by $I(0)$. \citet{Gray_2009} shows that explicitly solving for $i(s)$ requires use of Lambert's W-function. However, the relevant physical solutions may be attained by considering Equation \ref{eq:full_solution} in its limits.

When $i(s)\ll 1$, the maser is said to be in an {\it unsaturated} state. Taking this limit of Equation \ref{eq:full_solution} give
\begin{equation}
\label{eq:unsat_solution}
i(s) = i_0 \exp\left(s/L\right) + \beta\left[ \exp\left(s/L\right) - 1 \right]
\end{equation}
This shows the expected exponential growth of the intensity as the path length $s$ increases, the complete opposite of the exponential decay normally expected in radiative processes. Note that the inverse of the unsaturated growth length is analogous to the optical depth. As previously discussed, masers are self-limiting and will not remain in the unsaturated regime indefinitely. Once the population inversion becomes diminished, as shown in Equation \ref{eq:pop_inverse}, the maser will no longer experience exponential growth. When the initial flux is from a background source, the maser {\it gain} is $\ln \left( i(s)/i_0 \right)$ and will increase linearly with $s$ \citet{stahler_palla_2004}. Due to the exponential response of an unsaturated masers, they can display drastic time variability \citep{Elitzur_1992}. 

In the opposing limit of $i(s) \gg 1$, the maser is in the {\it saturated} state. The population inversion has been nearly depleted, and the linear terms in Equation \ref{eq:_full_solution} dominate,
\begin{equation}
\label{eq:sat_solution}
i(s) = \frac{s-s_{T}}{L}
\end{equation}
The transition to a saturated maser occurs at $s_T$, where $i(s)=1$. At that point, $i_0$ is negligible. Although the intensity growth of a saturated maser is linear, this regime represents the maximum efficiency that the pumping process can produce \citep{Elitzur_1992}. It represents the steady-state between the energy exhausted by stimulated emission and the energy provided by the pumping mechanism.

\subsection{Line Narrowing}
\label{sub:line_narrow}

Observations of masers have shown linewidths narrower than the expected thermal linewidths \citep[][e.g.,]{Elitzur_1992, stahler_palla_2004}. This is due to the exponential amplification in the unsaturated regime. Assume that the line follows a Doppler profile, so the absorption coefficient is expressed as
\begin{equation}
\label{eq:doppler_profile}
\alpha_\nu = \alpha_0 \exp\left( -\frac{(\nu - \nu_0)^2}{\Delta\nu_D^2} \right)
\end{equation}
The Doppler width $\Delta\nu_D$ is a function of the molecular mass and kinetic temperature \citep{stahler_palla_2004}. Using the ratio of the intensity difference between the central frequency $\nu_0$ and where the intensity falls to $e^{-1}$, the change in the linewidths is
\begin{equation}
\label{eq:line_narrow}
\Delta\nu^\star = \frac{\Delta\nu_D}{\sqrt{\alpha_0 s}}
\end{equation}
This result, as shown in \citet{stahler_palla_2004}, shows that the maser linewidth decreases with increasing path length. However, this result was derived using an unsaturated maser. A maser saturates from the inside out of it's line profile, since the greatest intensity is at the center. Thus, the center saturates before the wings of the profile do. The profile must then broaden, since the unsaturated wings are growing exponentially. This effect sets a limit on the narrowing of the line \citep{Elitzur_1992, stahler_palla_2004}. When the maser is completely saturated, it maintains a Doppler linewidth. 

\subsection{Beaming}
\label{sub:beaming}

The exponential growth of the maser also leads to a spatial narrowing of the observed maser source. This process is referred to as {\it beaming}. Figure \ref{fig:geometry} shows the effect of this beaming in a filamentary configuration. A complete solution for this geometry is shown in \citet{Elitzur_1991} A light ray enters from the right side, propagating through a saturated maser region (dark grey), towards the left. It hits an unsaturated region in the middle (white) where the the ray is beamed into a smaller peaked region. This occurs because this central region has the longest path length along the line-of-sight. The exponential increase with the path length leads to a significantly higher intensity ray in the center. Due to this, and the requirements needed for stimulated emission to occur, detected maser emission tends to be confined to very small regions. This result holds generally for other choices of the structure geometry \citep{Elitzur_1990_paperII}, where it can be shown that beaming is increased whenever the maser gain increases.

\subsection{Pumping Mechanisms}
\label{sub:pumping}

Stimulated emission requires an energy source to maintain the population inversion. The non-LTE conditions in the ISM are ideal for stimulated emission since they require the least amount of energy from a pump to create a population inversion. The pump mechanisms broadly fall into two categories: radiative and collisional. The environment and the characteristics of an energy transition also play a key role in how the pump mechanism achieves a population inversion. Table XXX ADD REF XXX shows typical properties of the regions where mega-maser emission is observed. This list is not exhaustive for masers in general, and the conditions for the same maser transitions in galactic sources can also differ (see \S XXX ADD SEC LABEL XXX). 

\subsubsection{Collisional Pumping}
\label{subsub:coll_pump}

Collisionally pumped (Type I) masers occur in higher-density regions, where the molecules responsible for the masers are excited from collisions primarily with H$_2$.  

\subsubsection{Radiative Pumping}
\label{subsub:rad_pump}

\begin{table} 
    \begin{tabular}{ c c c c c c }
        Molecule & Transition & Wavelength & E$_\mathrm{upp}$/k & $n$ & $T$ \\ 
         &  & (cm) & (K) & (cm$^{-3}$) & (K) \\ 
        OH & $^2\Pi_{3/2} (J=3/2; \Delta F = 0, \pm1)$ & 18 & 0.08 & $10^5 - 10^7$ & 100-200 \\ 
        H$_2$O & $6_{16} \longrightarrow 5_{23}$ & 1.35 & 640 & $10^7 - 10^9$ & 300-1000 \\ 
        H$_2$CO & $1_{10} \longrightarrow 1_{11}$ & 6.3 & 14 & $10^4 - 10^5$ & 20-40 \\ 
        SiO & $v=1; J=2 \longrightarrow 1$ & 0.35 & 1774 & $10^9 - 10^10$ & 700-1000 \\ 
    \end{tabular} 
    \caption{\label{tab:maser_props} Conditions for most observed mega-maser emitting species is shown. The values are primarily from \citet{stahler_palla_2004}. } 
\end{table}

