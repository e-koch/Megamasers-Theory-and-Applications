
Hydroxyl (OH) produces maser emission in its ground state (see Table \ref{subsub:rad_pump}). Two levels of splitting occur in this ground state: (1) the unclosed shell of electrons in the ground state gives rise to $\Lambda$-doubling of all rotational levels, and (2) hyperfine splitting between the unpaired electron and the H atom in the nucleus. Thus, there are four transitions in this ground-state with frequencies of 1612, 1665, 1667, and 1712 MHz with LTE line strengths of 1:5:9:1 \cite{lo2005}. The inversion mechanism accounting for all four transitions is complicated \citep{Elitzur_1992} and the relative maser line strengths can change drastically depending on the source's environment \citep{lo2005}. Masering in all four levels is extremely rare, and OH mega-maser detections are most often seen in the main 1665 \& 1667 MHz transitions. 

\section{Extragalactic Galactic Properties}
\label{sec:oh_gal_props}