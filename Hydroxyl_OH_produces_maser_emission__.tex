
Hydroxyl (OH) produces maser emission in its ground state (see Table \ref{subsub:rad_pump}). Two levels of splitting occur in this ground state: (1) the unclosed shell of electrons in the ground state gives rise to $\Lambda$-doubling of all rotational levels, and (2) hyperfine splitting occurs between the unpaired electron and the H atom in the nucleus. Thus, there are four transitions in this ground-state with frequencies of 1612, 1665, 1667, and 1712 MHz with LTE line strengths of 1:5:9:1 \cite{lo2005}. The inversion mechanism accounting for all four transitions is complicated \citep{Elitzur_1992} and the relative maser line strengths can change drastically depending on the source's environment \citep{lo2005}. Masering in all four levels is extremely rare, and OH mega-maser detections are most often seen in the main 1665 \& 1667 MHz transitions. 

\begin{itemize}
\item # detections
\item large scale surveys
\item galactic strength masers detectable out to ~20 Mpc w/ current instruments (Darling+12) 
\item trace highly obscured nuclear regions
\end{itemize}


\subsection{Extragalactic and Galactic Properties}
\label{sub:oh_gal_props}

\begin{itemize}
\item galactic sources are widespread
\item associated with dense molecular gas in star formation regions (cores, embedded IR sources, UCHII, outflows, H-H objects)
\item typically almost entirely circularly polarized galactic, not as much for mms (Gray text, Robishaw+08) 
\item typical linewidths 
\item missing strong 1665 emission
\item multi-velocity emission components
\item properties + relations determined in Darling paper III
\end{itemize}

\subsection{Galaxy Mergers \& Extreme Starbursts}

\begin{itemize}
\item LIRGS w/ and w/o mm emission (Lo); BGC 6240 examples if ULIRG w/o mm emission (Tacconi 1999)
\item role of IR radiation
\item Arecibo survey (Darling papers)
\item galactic weak anomalous OH maser emission (Mirable, Rodriquez, Ruiz 1989)
\item relation to H2CO emission - all non-masing merges show 6cm H2CO absorption, all masing show emission (Mangum+08)
\item ~100 pc extents from intense starbursts w/ high concentrations of molecular gas
\item always near to nucleus regions
\item 10^5 times more luminous than galactic GMCs (Downs, Solomon 1998)
\item may be explained by model of Parra et al. 2004
\item not necessarily amplification of bkg IR radiation
\item line profile wings explained as molecular outflows, starburst winds (Baan haschick 1987, baan haschick henkel 1989)
\item relation to CO luminosity (Darling+07) - OHM hosts break-away from the normal IR-CO correlation in SF galaxies (Gao solomon, 04)
\item seemingly unrelated to OH abundance  (Darling+12)
\item relation to IR spectra (Willett+2011a,b) - deep 10 um silicate absorption, steeper 20-30 um dust cont slope
\item limited time w/ ideal conditions during the merger?? - Darling+12
\item dust temp and optical depth significantly different b/w mm merger galaxies and those w/o - (Willett+11b)
\end{itemize}

\subsection{Luminosity function}

FIGURE 4 from Darling+ 02

\begin{itemize}
\item for 1667 from Arecibo OHM survey
\item consistent slope with function found for ULIRGs (Kim and Saunders 98)
\item study rate of SMBH mergers, and galaxy mergers in general -- constraints on grav wave bkg??
\item powerlaw form
\item corrections for Malmquist bias, etc...
\item will be detectable to z~4 w/ full SKA
\item GMRT could be capable of detecting dozens more
\item set constraints on galaxy formation during most active epochs (ie. z~2)
\item independent constraint on merger rates; accomanying those from optical w/ HST (Kim, Saunders 98)
\item extinction free redshift determination from sub-mm galaxies
\end{itemize}

\subsection{Zeeman Effect}

\begin{itemize}
\item McBride+14, mcbride heiles 13, robishaw heiles 09, robishaw+08
\item robishaw+08 measure splitting at 1667 in 5 ULIRGS w/ Arecibo+GBT (8 in survey)
\item B_LOS ~ .5 - 18 mG -- similar to galactic ones, suggesting local massive star formation is similar in starbursts
\item similar to B measured from synch radiation -- fields pervade multiple ISM phases
\item 
\end{itemize}

\subsection{Cosmology}