
Hydroxyl (OH) produces maser emission in its ground state (see Table \ref{subsub:rad_pump}). Two levels of splitting occur in this ground state: (1) the unclosed shell of electrons in the ground state gives rise to $\Lambda$-doubling of all rotational levels, and (2) hyperfine splitting occurs between the unpaired electron and the H atom in the nucleus. Thus, there are four transitions in this ground-state with frequencies of 1612, 1665, 1667, and 1712 MHz with LTE line strengths of 1:5:9:1 \cite{lo2005}. The inversion mechanism accounting for all four transitions is complicated \citep{Elitzur_1992} and the relative maser line strengths can change drastically depending on the source's environment \citep{lo2005}. Masering in all four levels is extremely rare, and OH mega-maser detections are most often seen in the main 1665 \& 1667 MHz transitions, with 1667 MHz emission often dominating for mega-masers. This is the opposite situation of galactic masers associated with HII regions, where the 1665 MHz line is the dominant feature in the spectrum. Supernovae remnants show significant emission in the 1712 MHz line. These differences point to differences in the environmental properties where the maser occurs. 

There are approximately 100 detected mega-masers to date (some detections have been called into question or cannot be adequately determined in the literature, see \citet{darling2002_paperIII}). The majority of these detections have come from large-scale surveys with explicit source selection. The most successful and complete of these surveys to date is that by \citet{darling_2000_paperI} with the Arecibo telescope, which resulted in 52 new OH mega-maser detections to $z<0.23$ \citep{darling2002_paperIII}. This survey, combined with previous detections over the previous few decades, have allowed the basic properties of the OH mega-masers and host galaxies to be fairly-well understood. These processes trace the environment within $\approx 100$ pc of the galactic nuclei in regions of extreme starbursts. These regions are typically highly obscured, making it difficult to study their nature at other wavelengths. Current instrumentation is capable of detecting galactic strength masers out to $\approx 20$ Mpc \citep{darling2012}, potentially allowing for the discovery of many more OH mega-masers in the near future.

\begin{itemize}
\item add blurb on all applications
\end{itemize}

\subsection{Extragalactic Properties}
\label{sub:oh_gal_props}

The dominance of 1665 MHz line emission over the 1667 MHz emission from OH mega-masers suggests they form in a different environment than OH galactic masers. Current detections support this, as nearly all OH mega-masers occur in luminous or ultra-luminous infrared galaxies (LIRGs/ULIRGs). These galaxies are unique since they harbour some of the most extreme starburst activity in the universe XXX CITE XXX. The extreme star formation rates lead to widespread heating of the dust and gas in the ISM through many giant HII regions, resulting in elevated emission in the IR. This connection of a dense, heated molecular gas environment to OH maser emission was recognized early on by \cite{Bottinelli_1987}, as such conditions were the observed locations of galactic OH maser emission. 

There are two plausible explanations for the production of OH mega-masers, both of which require fairly ordinary conditions \citep{lo2005}. The first occurs from low-gain, unsaturated masers which amplify continuum emission from the heated, diffuse gas around the galactic nucleus \citep[e.g.]{Baan_1985}. The second arises from compact regions which have high gains, leading to saturated maser emission \citep[e.g.,]{lonsdale2002}. Both types of emission have been observed, and \citet{Parra_2005} introduced a model capable of accounting for both types within a single gas phase.

\begin{itemize}
\item typically almost entirely circularly polarized galactic, not as much for mms (Gray text, Robishaw+08) 
\item typical linewidths 
\item missing strong 1665 emission
\item multi-velocity emission components
\item trace highly obscured nuclear regions
\item properties + relations determined in Darling paper III
\item time variability constraints - Darling paper IV
\item difference in polarization from OH galactic masers
\end{itemize}

\subsection{Galaxy Mergers \& Extreme Starbursts}

\begin{itemize}
\item LIRGS w/ and w/o mm emission (Lo); BGC 6240 examples if ULIRG w/o mm emission (Tacconi 1999)
\item role of IR radiation
\item Arecibo survey (Darling papers)
\item galactic weak anomalous OH maser emission (Mirable, Rodriquez, Ruiz 1989)
\item relation to H2CO emission - all non-masing merges show 6cm H2CO absorption, all masing show emission (Mangum+08)
\item ~100 pc extents from intense starbursts w/ high concentrations of molecular gas
\item always near to nucleus regions
\item 10^5 times more luminous than galactic GMCs (Downs, Solomon 1998)
\item may be explained by model of Parra et al. 2004
\item not necessarily amplification of bkg IR radiation
\item line profile wings explained as molecular outflows, starburst winds (Baan haschick 1987, baan haschick henkel 1989)
\item relation to CO luminosity (Darling+07) - OHM hosts break-away from the normal IR-CO correlation in SF galaxies (Gao solomon, 04)
\item seemingly unrelated to OH abundance  (Darling+12)
\item relation to IR spectra (Willett+2011a,b) - deep 10 um silicate absorption, steeper 20-30 um dust cont slope
\item limited time w/ ideal conditions during the merger?? - Darling+12
\item dust temp and optical depth significantly different b/w mm merger galaxies and those w/o - (Willett+11b)
\end{itemize}

\subsection{Luminosity function}

FIGURE 4 from Darling+ 02

\begin{itemize}
\item for 1667 from Arecibo OHM survey
\item consistent slope with function found for ULIRGs (Kim and Saunders 98)
\item study rate of SMBH mergers, and galaxy mergers in general -- constraints on grav wave bkg??
\item powerlaw form
\item corrections for Malmquist bias, etc...
\item will be detectable to z~4 w/ full SKA
\item GMRT could be capable of detecting dozens more
\item set constraints on galaxy formation during most active epochs (ie. z~2)
\item independent constraint on merger rates; accomanying those from optical w/ HST (Kim, Saunders 98)
\item extinction free redshift determination from sub-mm galaxies
\end{itemize}

\subsection{Zeeman Effect}

\begin{itemize}
\item McBride+14, mcbride heiles 13, robishaw heiles 09, robishaw+08
\item robishaw+08 measure splitting at 1667 in 5 ULIRGS w/ Arecibo+GBT (8 in survey)
\item B_LOS ~ .5 - 18 mG -- similar to galactic ones, suggesting local massive star formation is similar in starbursts
\item similar to B measured from synch radiation -- fields pervade multiple ISM phases
\item Equation 2 in robishaw+08 for B_LOS -- adapted Equation 1 in mcbride heiles 13
\item Some profiles show multiple components -- total of 14 individual components (spatial)
\item 1667 dominates over 1665, despite measuring similar B values to galactic -- maybe due to wider extragalactic maser lines
\item in Arp 220 - 4 spots - 2 w/ fields toward us and 2 w/ fields away
\item VLBI follow-up would give high-resolution field maps of B about the nucleus - robishaw heiles 09
\item additional 11 confident detections - mcbride heiles 13, 6.1-27.6 mG
\item new detections suggest larger than typical galactic B-fields -- median of 12 mG, ~2 larger than the Fish et al 2005
\item used all confirmed targets from Darling paper III and Willett 2012
\item Some non-detections may be due to much larger B, where the narrow line width approx doesn't hold
\item Can constrain some dynamics -- spherical cloud stability inferred suggests B plays a significant role in region dynamics, lws of ~20 km/s from Parra et al 05 in III Zw35 suggest non-gravitationally bound clouds. Magnetic support may keep confinement (not unlike some proposed schemes for cloud substructure)
\item McBride+14 include other B estimates from other methods, determine that ISM B-fields are greater in ULIRGs
\item dependent on minimum energy argument holding in the center of starburst galaxies
\item u_B > u_photon -- synchrotron cooling dominates over inverse Compton cooling -- CR electrons radiate energy via synchroton before leaving galaxy
\end{itemize}
