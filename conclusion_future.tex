\section{Conclusion \& Future Results}
\label{sec:conclusion}

Masers are known to be ubquituous throughout the Milky Way, and the increasing number of extra-galactic detections is beginning to confirm that for the rest of the nearby universe as well. 

In this report, I derive the basic principles of maser emission (\S\ref{sec:maser_theory}). Masers can exist in two states based on the population inversion of the transition: an unsaturated state with exponential gain with increasing path length, and a saturated state, where the gain increases linearly with the path length. Maser emission requires that (1) the population levels between two energy states must be inverted by an external energy source, the {\it pump}, and (2) the masing material must have velocity fluctuations within the thermal linewidth of the emission line in order for phase coherence to be maintained.

Mega-masers are aptly named as their inferred luminosities exceed those of galactic source by a factor $\gt 10^6$. Their high surface brightnesses and very compact nature make them ideal to study with Very Long Baseline interferometry, providing geometric distances of galaxies beyond 100 Mpc.

In  \S\ref{sec:OH}, I present the current state of research on OH mega-masers. Survey results show that the process is linked to a stage of major galaxy evolution, where extreme starburst activity is occuring near one of the nuclei in the system. OH mega-masers are also used to directly measure the magnetic fields in these extreme starburst regions. With upcoming improvements, predictions show that OH mega-masers can be used to probe the merger rates of galaxies up to $z\sim4$. Figure \ref{fig:oh_limits} shows this explicitly, where \citet{darling2002_lumfunc} have predicted the sensitivity limits of several new or future instruments (based on Equation \ref{eq:oh_lum}). 

I summarize the current efforts for using H$_2$O mega-masers to constrain fundamental physical properties in \S\ref{sec:h2o_mm}. Detections of H$_2$O mega-masers show that they arise from parsec-scales from AGN, and are only detected for certain classes of AGN, namely LINERs and Sy2 galaxies. H$_2$O mega-masers are used to measure SMBH masses to high precision, strengthening the evidence that SMBHs exist and to constrain the lower end of the $M_{\mathrm{BH}}-\sigma_{\star}$ relation. The direct detection of Keplerian circumnuclear disks about AGN allow for distances of galaxies past 100 Mpc to be measured. This provides an independent measure of the Hubble Constant, and with future work, is expected to reach uncertainty levels similar to other measurements.

Finally in \S\ref{sec:other}, I present detections of other mega-masing species, including the recent first detections of SiO and CH$_3$OH mega-masers. These species, particularly SiO, may potentially be used in tangent with the well-known mega-masers for similar applications.

Several new instruments and facilities have the potential to revolutionize this field in the very near future. Figure \ref{fig:oh_limits} shows the expected sensitivities of several instruments, some of which are now operational, for detecting OH mega-masers out to high redshifts. While more detections will continue to occur in the following years, the SKA is the most likely to make a profound effect in the mega-maser field. \citet{Morganti_2004} predict that the full-scale SKA will detect $10^5 - 10^6$ new OH and H$_2$O mega-masers. Such a significant increase in the number of detections would allow larger-scale population studies, and could revolutionize our understanding of galaxy mergers, star formation in extreme environments, and AGN.